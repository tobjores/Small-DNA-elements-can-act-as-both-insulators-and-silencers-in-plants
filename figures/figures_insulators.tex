% !TeX spellcheck = en_US
\documentclass[10pt]{article}

\usepackage{geometry}
\usepackage[no-math]{fontspec}
\usepackage{mathfont}
\usepackage{amsmath}
\usepackage[x11names, table]{xcolor}
\usepackage{pggfplot}
\usepackage{plantico}
\usepackage{tabularx}
\usepackage{xltabular}
\usepackage{booktabs}
\usepackage{dnaseq}
\usepackage{caption}
\usepackage{newfloat}
\usepackage{xspace}
\usepackage[hidelinks]{hyperref}
\usepackage{polyglossia}
\setdefaultlanguage[variant = american]{english}
\usepackage[noabbrev]{cleveref}

%%% font definitions
\defaultfontfeatures{Ligatures = TeX}
\setsansfont{Arial}
\renewcommand{\familydefault}{\sfdefault}
\mathfont{Arial}
\newfontfamily{\signiffont}{Arial Unicode MS}
\newfontfamily{\symbolfont}{Symbol}

%%% font sizes
\newcommand{\fignormal}{\scriptsize} % 7pt
\newcommand{\figsmall}{\fontsize{6}{7}\selectfont} % 6pt
\newcommand{\figtiny}{\tiny} % 5pt
\newcommand{\figlarge}{\footnotesize} % 8pt
\newcommand{\fighuge}{\small} % 9pt

%%% page layout
\geometry{letterpaper, textwidth = 180mm, textheight = 235mm, marginratio = 1:1}
\renewcommand{\textfraction}{0}
\makeatletter
	\setlength{\@fptop}{0pt}
\makeatother
    
%%% header and footer
\pagestyle{empty}

%%% no indentation
\setlength\parindent{0pt}

%%% table layout
\renewcommand{\arraystretch}{1.33}

\setlength{\aboverulesep}{0pt}
\setlength{\belowrulesep}{0pt}

\renewcommand{\tabularxcolumn}[1]{m{#1}}

%%% useful lengths
\newlength{\templength}

\setlength{\columnsep}{4mm}
\newlength{\twocolumnwidth}
\setlength{\twocolumnwidth}{0.5\textwidth}
\addtolength{\twocolumnwidth}{-.5\columnsep}
\newlength{\threecolumnwidth}
\setlength{\threecolumnwidth}{.333\textwidth}
\addtolength{\threecolumnwidth}{-.666\columnsep}
\newlength{\twothirdcolumnwidth}
\setlength{\twothirdcolumnwidth}{2\threecolumnwidth}
\addtolength{\twothirdcolumnwidth}{\columnsep}
\newlength{\fourcolumnwidth}
\setlength{\fourcolumnwidth}{.25\textwidth}
\addtolength{\fourcolumnwidth}{-.75\columnsep}
\newlength{\threequartercolumnwidth}
\setlength{\threequartercolumnwidth}{3\fourcolumnwidth}
\addtolength{\threequartercolumnwidth}{2\columnsep}

%%% tikz setup
\pgfmathsetseed{928}

%%% general
\usetikzlibrary{calc, positioning, arrows.meta, arrows, bending, external, backgrounds, topaths, shapes.arrows, shapes.geometric, shapes.symbols, decorations.markings}

\tikzexternalize[prefix = extFigures/, only named = true]

%%% layers
\pgfdeclarelayer{background}
\pgfdeclarelayer{semi foreground}
\pgfdeclarelayer{foreground}
\pgfsetlayers{background, main, semi foreground, foreground}


%%% colors
\colorlet{35Spromoter}{DarkSeaGreen3}
\colorlet{35Senhancer}{DodgerBlue1}


%%% save and use jpeg for externalization
\tikzset{
	% Defines a custom style which generates BOTH, .pdf and .jpg export
	% but prefers the .jpg on inclusion.
	jpeg export/.style = {
		external/system call/.add = {}{%
			&& pdftocairo -jpeg -r #1 -singlefile "\image.pdf" "\image" %
		},
		/pgf/images/external info,
		/pgf/images/include external/.code = {%
			\includegraphics[width=\pgfexternalwidth,height=\pgfexternalheight]{##1.jpg}%
		},
	},
	jpeg export/.default = 300
}


%%% calculate distance between two tikz coordinates
% use: \distance{<coordinate 1>}{<coordinate 2>}
\makeatletter
	\newcommand{\distance}[2]{
		\path (#1);
		\pgfgetlastxy{\xa}{\ya} 
		\path (#2);
		\pgfgetlastxy{\xb}{\yb}   
		\pgfpointdiff{\pgfpoint{\xa}{\ya}}{\pgfpoint{\xb}{\yb}}%
		\setlength{\xdistance}{\pgf@x}
		\setlength{\ydistance}{\pgf@y}
	}
	
	\def\convertto#1#2{\strip@pt\dimexpr #2*65536/\number\dimexpr 1#1\relax\,#1}
\makeatother

\newlength{\xdistance}
\newlength{\ydistance}


%%% useful styles
\tikzset{
	shorten both/.style = {shorten < = #1, shorten > = #1}
}


%%% set raw data directory
\pggfdatadir{rawData/}


%%% set pggfplot options
\newlength{\groupplotsep}
\setlength{\groupplotsep}{0.75mm}

\pggfset{
	xlabel height = 7.85mm,
	ylabel width = 9.35mm,
	anchor = outer north west,
	thin,
	unbounded coords = jump,
	axis background/.style = {fill = white},
	label style = {node font = \fignormal, text depth = 0pt, align = center},
	title font = \figlarge,
	facet sep = \groupplotsep,
	tickwidth = \groupplotsep,
	inner ticks,
	tick label style = {node font = \figsmall, /pgf/number format/fixed},
	tick align = outside,
	tick pos = lower,
	every x tick label/.append style = {inner xsep = 0pt},
	every y tick label/.append style = {inner ysep = 0pt},
	every tick/.append style = {black, thin},
	max space between ticks = 20,
	scaled ticks = false,
	legend style = {
		node font = \figsmall,
		fill = none,
		draw = none,
		/tikz/every even column/.append style = {column sep = \groupplotsep},
		row sep = -1pt
	},
	legend cell align = left,
	mark size = 1,
	boxplot/whisker extend = {\pgfkeysvalueof{/pgfplots/boxplot/box extend} * 0.5},
	boxplot/every median/.style = thick,
	sample size = {node font = \figtiny},
	plot/diagonal/.default = {gray, dashed},
	pggf colorbar style = {
		yticklabel style = {node font = \figtiny},
		title style = {node font = \figsmall, inner sep = 0pt},
		tick align = outside,
		tick pos = lower,
		tickwidth = \groupplotsep
	},
}

\pgfplotsset{
	every non boxed x axis/.style = {},
	every non boxed y axis/.style = {},
	xytick/.style = {xtick = {#1}, ytick = {#1}},
	empty legend/.style={/pgfplots/legend image code/.code={\path (.3cm, 0pt);}},
	quiver arrow/.style = {thick, {-Stealth[scale = .5]}}
}


%%% variant of /tikz/fill that accepts 'none' as a color (`fill = none` disables filling, `fill* = none` uses color 'none' for filling)
\makeatletter
	\tikzset{
		fill*/.code = {
      \tikz@addoption{\pgfsetfillcolor{#1}}%
      \def\tikz@fillcolor{#1}%
    	\tikz@addmode{\tikz@mode@filltrue}%
		}
	}
\makeatother


%%% set default plot options
\pgfqkeys{/pggf/plot}{
	bar_defaults/.append style = {fill opacity = .5},
	boxplot_defaults/.append style = {fill opacity = .5, boxplot/draw/average/.code = {}},
	violin_defaults/.append style = {fill opacity = .5, boxplot options = {boxplot/draw/average/.code = {}}},
	hline_defaults/.append style = {densely dotted},
	trendline_defaults/.append style = {Sienna1, dashed}
}


%%% custom legends
\pgfplotsset{
	/pgfplots/ybar legend/.style = {
		/pgfplots/legend image code/.code = {
			\draw [##1, /tikz/.cd, bar width = 2pt, yshift = -0.3em, bar shift = 0pt]
			plot coordinates {
				(0em, 0.8em)
				(\pgfplotbarwidth, 1.5em)
				(2*\pgfplotbarwidth, 0.6em)
			};
		},
	},
	/pgfplots/xbar legend/.style = {
		/pgfplots/legend image code/.code = {
			\draw [##1, /tikz/.cd, bar width = 2pt, yshift = -0.2em, bar shift = 0pt]
			plot coordinates {
				(0.8em, 0em)
				(1.5em, \pgfplotbarwidth)
				(0.6em, 2*\pgfplotbarwidth)
			};
		},
	},
}

%%% useful commands
\renewcommand{\textprime}{\char"2032{}}
\newcommand{\textalpha}{\char"03B1{}}
\renewcommand{\textbeta}{\char"03B2{}}
\newcommand{\textDelta}{\char"0394{}}
\newcommand{\textlambda}{\char"03BB{}}
\renewcommand{\textle}{\char"2264{}}
\renewcommand{\textge}{\char"2265{}}

\newcommand{\enrichment}[1][]{%
	\if\relax#1\relax%
		\def\enrSuffix{}%
	\else%
		\def\enrSuffix{, #1}%
	\fi%
	$\log_2$(enrichment\enrSuffix)%
}
\newcommand{\dlratio}[1][]{%
	$\log_2$(NanoLuc/Luc)%
}

%% define shorthands for enhancers, species, insulators, ...
% use: \defname[<color model>]{<shorthand>}{<text form>}{<color>}
% sets both the text form (obtained with \usename{<shorthand>}) and the color
\newcommand{\defname}[4][HTML]{%
	\expandafter\def\csname#2\endcsname{#3\xspace}%
	\definecolor{#2}{#1}{#4}%
}

\newcommand{\usename}[1]{\csname#1\endcsname}

% define enhancers
\defname[named]{none}{\vphantom{A}none}{black}
\defname[named]{35S}{35S}{35Senhancer}
\defname[named]{AB80}{\textit{AB80}}{OrangeRed1}
\defname[named]{Cab-1}{\textit{Cab-1}}{Gold3}

% define species
\defname[named]{tobacco}{\textit{N. benthamiana}}{Green4}
\defname[named]{maize}{maize}{Goldenrod2}
\defname[named]{rice}{rice}{Khaki1}
\defname[named]{Arabidopsis}{Arabidopsis}{SeaGreen4}

% define species combos for plot titles
\newcommand{\defspeciescombo}[2]{
	\expandafter\def\csname #1/#2\endcsname{\usename{#1} vs. \usename{#2}}
	\tikzset{
		#1/#2/.style = {left color = #1!20, right color = #2!20}
	}
}

\defspeciescombo{tobacco}{maize}
\defspeciescombo{Arabidopsis}{tobacco}\expandafter\def\csname Arabidopsis/tobacco\endcsname{\usename{Arabidopsis} vs. \textit{N. benth.}}
\defspeciescombo{rice}{maize}
\defspeciescombo{maize}{maize}

% define insulators
\defname[named]{noEnh}{noEnh}{black}
\defname[named]{noIns}{noIns}{35Senhancer}
\defname{beta-phaseolin}{\textbeta-phaseolin}{CC79A7}
\defname{TBS}{TBS}{D55E00}
\defname{lambda-EXOB}{$\lambda$-EXOB}{0072B2}
\defname{BEAD-1C}{BEAD-1C}{D6C700}
\defname{UASrpg}{UASrpg}{009E73}
\defname{sIns1}{sIns1}{56B4E9}
\defname{sIns2}{sIns2}{E69F00}
\defname[named]{gypsy}{gypsy}{gray}

\newcommand{\deffrag}[3]{
	\defname[named]{#1_#2_#3}{\usename{#1}\\[-.25\baselineskip]\figtiny(#2\textendash #3)}{#1}
}

\deffrag{beta-phaseolin}{1395}{1564}
\deffrag{TBS}{756}{925}
\deffrag{lambda-EXOB}{1}{170}
\deffrag{BEAD-1C}{246}{415}
\deffrag{UASrpg}{1}{170}
\deffrag{gypsy}{54}{223}
\deffrag{beta-phaseolin}{781}{612}\colorlet{beta-phaseolin_781_612}{beta-phaseolin!50}
\deffrag{beta-phaseolin}{2356}{2187}\colorlet{beta-phaseolin_2356_2187}{beta-phaseolin!50!black}
\deffrag{TBS}{588}{757}
\deffrag{lambda-EXOB}{666}{497}
\deffrag{sIns2}{335}{504}

% maize samples
\defname[named]{R1_husk}{husk (R1)}{Goldenrod1}
\defname[named]{R1_leaf}{leaf (R1)}{Goldenrod1}\colorlet{R1_leaf}{Goldenrod1!83!Goldenrod4}
\defname[named]{R1_silk}{silk (R1)}{Goldenrod1}\colorlet{R1_silk}{Goldenrod1!67!Goldenrod4}
\defname[named]{R1_stalk}{stalk (R1)}{Goldenrod1}\colorlet{R1_stalk}{Goldenrod1!50!Goldenrod4}
\defname[named]{V6_leaf}{leaf (V6)}{Goldenrod1}\colorlet{V6_leaf}{Goldenrod1!33!Goldenrod4}
\defname[named]{V6_root_tip}{root tip (V6)}{Goldenrod1}\colorlet{V6_root_tip}{Goldenrod1!17!Goldenrod4}
\defname[named]{V6_root_mature}{root mid (V6)}{Goldenrod4}

% fragment combinations
\defname[named]{D2}{D2}{beta-phaseolin}
\defname[named]{T30}{T30}{TBS}\colorlet{T30}{TBS!50}
\defname[named]{T21}{T21}{TBS}\colorlet{T21}{TBS!65}
\defname[named]{T27}{T27}{TBS}\colorlet{T27}{TBS!80}
\defname[named]{T32}{T32}{TBS}\colorlet{T32}{TBS!95}
\defname[named]{T24}{T24}{TBS}\colorlet{T24}{TBS!95!black}
\defname[named]{T25}{T25}{TBS}\colorlet{T25}{TBS!80!black}
\defname[named]{T19}{T19}{TBS}\colorlet{T19}{TBS!65!black}
\defname[named]{T9}{T9}{TBS}\colorlet{T9}{TBS!50!black}
\defname[named]{T9+D2}{T9+D2}{UASrpg}\colorlet{T9+D2}{UASrpg!75}
\defname[named]{T32+D2}{T32+D2}{UASrpg}
\defname[named]{T27+D2}{T27+D2}{UASrpg}\colorlet{T27+D2}{UASrpg!75!black}

% misc names
\defname[named]{with}{with \usename{35S} enh}{35Senhancer}
\defname[named]{without}{without \usename{35S} enh}{gray}
\defname[named]{insulator}{insulator}{Gold3}
\defname[named]{silencer}{silencer}{Brown2}
\defname[named]{controls}{controls}{gray}
	
\expandafter\def\csname STARRseq\endcsname{Plant STARR-seq}
\expandafter\def\csname empty legend\endcsname{\relax}

%%% declare figure types
\newif\ifnpc

% main figures
\newif\ifmain
\newcounter{fig}

\addto\captionsenglish{\renewcommand\figurename{Figure}}

\newenvironment{fig}{%
	\begin{figure}[p]%
		\stepcounter{fig}%
		\pdfbookmark{\figurename\ \thefig}{figure\thefig}%
		\tikzsetnextfilename{figure\thefig}%
		\fignormal%
		\centering%
}{%
	\end{figure}%
	\clearpage%
	\ifnpc%
		\makenextpagecaption%
		\global\npcfalse%
	\fi%
}

% supplementary figures
\newif\ifsupp
\newcounter{sfig}

\DeclareFloatingEnvironment[fileext = losf, name = Supplementary \figurename]{suppfigure}

\newenvironment{sfig}{%
	\begin{suppfigure}[p]%
		\stepcounter{sfig}%
		\setcounter{subfig}{0}%
		\pdfbookmark{\suppfigurename\ S\thesfig}{supp_figure\thesfig}%
		\tikzsetnextfilename{supp_fig\thesfig}%
		\fignormal%
		\centering%
}{%
	\end{suppfigure}%
	\clearpage%
	\ifnpc%
		\makenextpagecaption%
		\global\npcfalse%
	\fi%
}

% supplementary tables
\newif\ifstab
\newcounter{stab}

\addto\captionsenglish{\renewcommand\tablename{Supplementary Table}}

\newenvironment{stab}{%
	\begin{table}[p]%
		\stepcounter{stab}%
		\pdfbookmark{\tablename\ S\thestab}{supp_table\thestab}%
		\small%
}{%
	\end{table}%
	\clearpage%
}

\newenvironment{longstab}{%
	\begingroup%
	\stepcounter{stab}%
	\pdfbookmark{\tablename\ S\thestab}{supp_table\thestab}%
	\small%
}{%
	\endgroup%
	\clearpage%
}

% TOC icon
\newif\ifTOCicon

\newenvironment{TOCicon}{%
		\pdfbookmark{TOC icon}{TOC_icon}%
		\tikzsetnextfilename{TOC_icon}%
		\fignormal%
		\centering%
}{%
	\clearpage%
}

%% command to put the caption on the next page
% use instead of a normal caption: \nextpagecaption{<caption text>}
\makeatletter
	\newcommand{\nextpagecaption}[1]{%
		\global\npctrue%
		\xdef\@npctype{\@currenvir}%
		\captionlistentry{#1}%
		\long\gdef\makenextpagecaption{%
				\csname\@npctype\endcsname%
					\ContinuedFloat%
					\caption{#1}%
				\csname end\@npctype\endcsname%
				\clearpage%
		}%
	}
\makeatother

%%% caption format
\DeclareCaptionJustification{nohyphen}{\hyphenpenalty = 10000}

\captionsetup{
	labelsep = period,
	justification = nohyphen,
	singlelinecheck = false,
	labelfont = {bf},
	font = small,
	figureposition = below,
	tableposition = above
}
\captionsetup[figure]{skip = .5\baselineskip}

\newcommand{\titleend}{. }
\newcommand{\nextentry}{ }
\newcommand{\captiontitle}[2][]{#2#1\titleend}

%%% subfigure labels
\newif\ifsubfigupper
\subfiguppertrue

\newcounter{subfig}[figure]

\tikzset{
	subfig label/.style = {anchor = north west, inner sep = 0pt, font = \normalsize\bfseries}
}

\newcommand{\subfiglabel}[2][]{
	\node[anchor = north west, inner sep = 0pt, font = \large\bfseries, #1] at (#2) {\strut\stepcounter{subfig}\ifsubfigupper\Alph{subfig}\else\alph{subfig}\fi};
}

\newcommand{\subfigrefsep}{\textbf{)}}
\newcommand{\subfigrefand}{, }
\newcommand{\subfigrefrange}{\textendash}

\newcommand{\subfigunformatted}[1]{\ifsubfigupper\uppercase{#1}\else\lowercase{#1}\fi}
\newcommand{\plainsubfigref}[1]{\textbf{\subfigunformatted{#1}}}
\newcommand{\subfig}[1]{\textbf{\plainsubfigref{#1}}\subfigrefsep}
\newcommand{\subfigtwo}[2]{\textbf{\plainsubfigref{#1}\subfigrefand\plainsubfigref{#2}}\subfigrefsep}
\newcommand{\subfigrange}[2]{\textbf{\plainsubfigref{#1}\subfigrefrange\plainsubfigref{#2}}\subfigrefsep}
\newcommand{\parensubfig}[2][]{(#1\plainsubfigref{#2})}
\newcommand{\parensubfigtwo}[3][]{(#1\plainsubfigref{#2}\subfigrefand\plainsubfigref{#3})}
\newcommand{\parensubfigrange}[3][]{(#1\plainsubfigref{#2}\subfigrefrange\plainsubfigref{#3})}

\newcommand{\supports}[1]{ (Supports #1)}

%%% cross-reference setup
\newcommand{\crefrangeconjunction}{\subfigrefrange}
\newcommand{\crefpairgroupconjunction}{\subfigrefand}
\newcommand{\crefmiddlegroupconjunction}{, }
\newcommand{\creflastgroupconjunction}{, }

\crefname{figure}{Figure}{Figures}
\crefname{suppfigure}{Supplementary Figure}{Supplementary Figures}
\crefname{table}{Supplementary Table}{Supplementary Tables}

%%% what to include
\maintrue
\supptrue
\stabtrue
\TOCicontrue

%%%%%%%%%%%%%%%%%%%%%%%%%%%%%%%%%%%%
%%%   ||    preamble end    ||   %%%
%%%--\\//------------------\\//--%%%
%%%   \/   begin document   \/   %%%
%%%%%%%%%%%%%%%%%%%%%%%%%%%%%%%%%%%%

\begin{document}
	\sffamily
	\frenchspacing
	
	%%% change numbering of supplementary figures and tables
	\renewcommand\thesuppfigure{S\arabic{suppfigure}}
	\renewcommand\thetable{S\arabic{table}}
	
	%%% Main figures start
	\ifmain
		
		\begin{fig}
			\begin{tikzpicture}

	%%% scheme
	\coordinate (scheme) at (0, 0);
	
	% insulators
	\node[anchor = north, xshift = .5\fourcolumnwidth, yshift = -.5\columnsep] (lins) at (scheme) {known insulators};
	
	\draw[line width = .1cm, Gold1] ($(scheme |- lins.base) + (.6, -.2)$) -- ++(.25\fourcolumnwidth, 0) coordinate (c1);
	\draw[line width = .1cm, Goldenrod2] (c1) ++(.3, 0) -- ++(.75\fourcolumnwidth - 1.5cm, 0);
	\draw[line width = .1cm, Goldenrod1] ($(scheme |- c1) + (.4, -.25)$) -- ++(.4\fourcolumnwidth, 0) coordinate (c2);
	\draw[line width = .1cm, Gold2] (c2) ++(.2, 0) -- ++(.6\fourcolumnwidth - 1cm, 0);
	\draw[line width = .1cm, insulator] ($(scheme |- c2) + (.2, -.25)$) coordinate (c3) -- ++(\fourcolumnwidth - .4cm, 0) coordinate (c4);
	
	\distance{c3}{c4}
	
	\coordinate (frags) at ($(scheme |- c3) + (.1, -.65)$);
	
	\node[anchor = south] at (lins |- frags) {split into smaller fragments};
	
	\foreach \x in {0, ..., 3} {
		\draw[line width = .1cm, insulator] (frags) ++({\x * (.25\xdistance + .2cm/3)}, 0) -- ++(.25\xdistance, 0);
		\ifnum\x>0
			\draw[line width = .1cm, insulator] (frags) ++({\x * (.25\xdistance + .2cm/3)}, 0) ++(-.125\xdistance - .2cm/6, -.15) coordinate (f\x) -- ++(.25\xdistance, 0);
		\fi
	}
	
	\begin{pgfonlayer} {background}
		\draw[thin, gray, out = -90, in = 90] (c3) ++(.5\pgflinewidth, 0) to (frags) (c4) to ($(frags) + (\xdistance + .2cm - .5\pgflinewidth, 0)$);
	\end{pgfonlayer}
	
	\draw[Bar-Bar] (f3) ++(0, -.175) -- ++(.25\xdistance, 0) node[pos = .5, anchor = north, node font = \figsmall] {170 bp};
	
	% construct
	\coordinate[shift = {({.5 * (\fourcolumnwidth - 3.3cm)}, -1.33)}] (construct) at (scheme |- frags);
	
	\draw[line width = .2cm, -Triangle Cap, 35Senhancer]  ($(construct) + (.2, 0)$) -- ++(.6, 0) coordinate[pos = .5] (enh);

	\draw[Latex-] (enh) ++(0, -.15) -- ++(0, -.4) coordinate (lenh);
	\node[anchor = north] at (lenh -| .5\fourcolumnwidth, 0) {enhancer: \textcolor{35S}{\bfseries\usename{35S}}, \textcolor{AB80}{\bfseries\usename{AB80}}, or \textcolor{Cab-1}{\bfseries\usename{Cab-1}}};

	\fill[insulator] (construct) ++(.9, -.2) -- ++(60:.5) coordinate (ins) -- ++(-60:.5) -- cycle;

	\draw[line width = .2cm, 35Spromoter] (construct) ++(1.5, 0) -- ++(.3, 0) coordinate[pos = .5] (pro);
	\draw[-{Stealth[round]}, semithick] (pro) ++(.15cm + .5\pgflinewidth, 0) |- ++(.35, .3);
	
	\node[draw = black, thin, anchor = west, minimum width = 1cm, outer xsep = 0pt] (GFP) at ($(pro) + (.45, 0)$) {GFP};
	
	\begin{pgfonlayer} {background}
		\fill[Orchid1] (GFP.north west) ++(.15, 0) rectangle ($(GFP.south west) + (.05, 0)$);
	\end{pgfonlayer}
	
	\coordinate[xshift = .2cm] (construct end) at (GFP.east);
	
	\begin{pgfonlayer} {background}
		\draw[thick, {Bar[width = .1cm]}-] (construct) -- (GFP.west);
		\draw[thick, -{Bar[width = .1cm]}] (GFP.east) -- (construct end);
	\end{pgfonlayer}
	
	\draw[thick, -Latex] (f2) ++(.125\xdistance, -.1) to[out = -90, in = 90] ($(ins) + (0, .05)$);
	

	%%% enrichment insulator fragments
	\coordinate (iLib) at (scheme -| \textwidth - \threequartercolumnwidth, 0);
	
	\begin{pggfplot}[%
		at = {(iLib)},
		total width = {\textwidth - \fourcolumnwidth - \threecolumnwidth - 2\columnsep},
		xlabel = enhancer,
		ylabel = \enrichment,
		xticklabels are csnames,
		xticklabel style = {font = \vphantom{\usename{AB80}}},
		title is csname,
		title color from name,
		enlarge y limits = {lower, value = 0.1},
	]{iLib_enrichment}
	
		\pggf_annotate(zeroline)
	
		\pggf_violin(
			style from column = {fill* = xticklabels},
			extra mark = {mark size = 2.5}{noIns}
		)
	
	\end{pggfplot}
	
	
	%%% iLib correlation assay systems
	\coordinate (species) at (scheme -| \textwidth - \threecolumnwidth, 0);
	
	\begin{pggfplot}[%
		at = {(species)},
		total width = \threecolumnwidth,
		xlabel = {\textbf{\tobacco:} \enrichment},
		ylabel = {\textbf{\maize:} \enrichment},
		col title = tobacco,
		row title = maize,
		title is csname,
		title color from name,
		xytick = {-10, ..., 10},
		enlarge y limits = {upper, value = .05}
	]{iLib_cor_species}
	
		\pggf_scatter()
	
		\pggf_stats(stats = {n,spearman,rsquare})
	
	\end{pggfplot}
	
	
	%%% enrichment by position
	\coordinate[yshift = -\columnsep] (position) at (scheme |- xlabel.south);

	\begin{pggfplot}[%
		at = {(position)},
		yshift = -.67cm,
		total width = \textwidth,
		xlabel = position,
		ylabel = \enrichment,
		title is csname,
		title color from name,
		xticklabel style = {font = \vphantom{\usename{lambda-EXOB}}, rotate = 45, anchor = north east},
		tight y,
		xtick = {1, 300, 600, ..., 5000},
		ytick = {-10, ..., 10},
		col titles from table = {save = pggfcoltitles}
	]{iLib_position}
	
		\pggf_annotate(zeroline)
		
		\pggf_annotate(
			only facet = 1,
			annotation = {
				\fill[\pggfcolname, opacity = .1] (750, \pggfymin) rectangle (1950, \pggfymax) (2250, \pggfymin) rectangle (2500, \pggfymax) (2700, \pggfymin) rectangle (3250, \pggfymax);
				\draw[thin, gray, x radius = .18 cm, y radius = .09cm] (1401.5, 4.00153919712402) ellipse[] (1401.5, 2.3572122522011503) ellipse[];
			}
		)
		
		\pggf_annotate(
			only facet = 2,
			annotation = {
				\fill[\pggfcolname, opacity = .1] (\pggfxmin, \pggfymin) rectangle (750, \pggfymax) (1150, \pggfymin) rectangle (\pggfxmax, \pggfymax);
			}
		)
		
		\pggf_annotate(
			only facet = 3,
			annotation = {
				\fill[\pggfcolname, opacity = .1] (\pggfxmin, \pggfymin) rectangle (600, \pggfymax);
				\draw[thin, gray, x radius = .18 cm, y radius = .09cm] (913.5, 1.8684711422889908) ellipse[] (913.5, 3.238528677573467) ellipse[];
			}
		)

		\pggf_annotate(
			only facet = 9,
			annotation = {
				\fill[\pggfcolname, opacity = .1] (1200, \pggfymin) rectangle (1750, \pggfymax);
				\draw[thin, gray, x radius = .18 cm, y radius = .09cm] (2904.5, 0.995524601350010) ellipse[] (2904.5, 1.9804632898609085) ellipse[];
			}
		)

		\pggf_annotate(
			only facet = 10,
			annotation = {
				\fill[\pggfcolname, opacity = .1] (\pggfxmin, \pggfymin) rectangle (400, \pggfymax);
				\draw[thin, gray, x radius = .18 cm, y radius = .09cm] (1429.5, 2.272147967155985) ellipse[] (1429.5, 0.7167029580998092) ellipse[];
			}
		)

		\pggf_annotate(
			only facet = 11,
			annotation = {
				\fill[\pggfcolname, opacity = .1] (250, \pggfymin) rectangle (600, \pggfymax);
			}
		)

		\pggf_annotate(
			only facet = 12,
			annotation = {
				\fill[\pggfcolname, opacity = .1] (150, \pggfymin) rectangle (400, \pggfymax);
			}
		)
		
		\pggf_hline()
	
		\pggf_quiver(
			color from col name,
			quiver arrow
		)
	
	\end{pggfplot}
	
	% invert list of coltitles
	\def\pggftitlesinv{}
	\expandafter\pgfplotsinvokeforeach\expandafter{\pggfcoltitles}{
		\if\relax\pggftitlesinv\relax
			\def\pggftitlesinv{#1}
		\else
			\preto\pggftitlesinv{#1,}
		\fi
	}
	
	% add coltitles
	\coordinate[xshift = \groupplotsep] (title0) at (pggfplot c8r1.east |- position);
	
	\foreach \coltitle [count = \i, remember = \i as \lasti initially 0, evaluate = \i as \invi using int(9-\i), expand list] in {\pggftitlesinv} {
		\ifnum\invi>1\relax
			\distance{pggfplot c\invi r1.west}{pggfplot c\invi r1.east}
		\else
			\distance{pggfplot c\invi r1.west}{title\lasti.west}
			\addtolength{\xdistance}{-\groupplotsep}
		\fi
		\node[anchor = north east, xshift = -\groupplotsep, draw, fill = \coltitle!20, outer sep = 0pt, text depth = .15\baselineskip, minimum width = \xdistance, font = \figlarge] (title\i) at (title\lasti.north west) {\vphantom{\usename{lambda-EXOB}}\usename{\coltitle}};
		\begin{pgfonlayer}{background}
			\draw[gray, top color = \coltitle!20, bottom color = white, thin, line join = bevel] (title\i.south west) -- (pggfplot c\invi r1.north west) -- (pggfplot c\invi r1.north east) -- (title\i.south east);
		\end{pgfonlayer}
	}
	
	
	%%% iLib correlation GC
	\coordinate[yshift = -\columnsep] (GC) at (scheme |- xlabel.south);
	
	\begin{pggfplot}[%
		at = {(GC)},
		total width = \twocolumnwidth,
		xlabel = GC content,
		ylabel = \enrichment,
		title is csname,
		title color from name
	]{iLib_cor_GC}
		
		\pggf_annotate(zeroline)
	
		\pggf_hline()
	
		\pggf_scatter(
			color from col name,
			opacity = .5
		)
	
		\pggf_stats(
			stats = {n,spearman,rsquare},
			text mark style = {
				anchor = south west,
				at = {(current axis.left of origin)}
			},
			only facets = 1
		)
		\pggf_stats(
			stats = {n,spearman,rsquare},
			only facets = 2
		)
	
	\end{pggfplot}
	
	
	%%% correlation across enhancers
	\coordinate (enh cor) at (GC -| \textwidth - \twocolumnwidth, 0);
	
	\begin{pggfplot}[%
		at = {(enh cor)},
		total width = \twocolumnwidth,
		xlabel = {\textbf{\usename{AB80} or \usename{Cab-1}:} \enrichment},
		ylabel = {\textbf{\usename{35S}:} \enrichment},
		row title = 35S,
		title is csname,
		title color from name,
		enlarge y limits = {upper, value = .1},
		xytick = {-10, -9, ..., 10}
	]{iLib_cor_enh}
	
		\pggf_scatter(
			data = iLib_cor_enh_fragments,
			color from col name,
			opacity = .5
		)
		
		\pggf_scatter(
			data = iLib_cor_enh_control,
			mark size = 2.5
		)
		
		\pggf_stats(stats = {n,spearman,rsquare})
	
		\pggf_abline(gray, dashed)
	
	\end{pggfplot}
	

	%%% subfigure labels
	\subfiglabel{scheme}
	\subfiglabel{iLib}
	\subfiglabel{species}
	\subfiglabel{position}
	\subfiglabel{GC}
	\subfiglabel{enh cor}

\end{tikzpicture}%
			\nextpagecaption{%
				\captiontitle{Short fragments exhibit enhancer-blocking insulator activity}%
				\subfig{A} Known insulators were split into partially overlapping 170-bp fragments. The insulator fragments were cloned in the forward or reverse orientation between a \usename{35S}, \usename{AB80}, or \usename{Cab-1} enhancer and a 35S minimal promoter (green rectangle) driving the expression of a barcoded GFP reporter gene. Constructs without an enhancer (none) but with insulator fragments were also created.\nextentry
				\subfig{B} All insulator fragment constructs were pooled and subjected to Plant STARR-seq in \tobacco leaves (\tobacco) and \maize protoplasts (\maize). Reporter mRNA enrichment was normalized to a control construct without an enhancer or insulator (noEnh; log2 set to 0). The enrichment of a control construct without an insulator is indicated as a black dot. Violin plots represent the kernel density distribution and the box plots inside represent the median (center line), upper and lower quartiles, and 1.5$\times$ interquartile range (whiskers) for all corresponding constructs. Numbers at the bottom of each violin indicate the number of samples in each group.\nextentry
				\subfig{C} Correlation between the enrichment of insulator fragments in constructs with the \usename{35S} enhancer in \tobacco leaves and \maize protoplasts.\nextentry
				\subfig{D} Enrichment of constructs with insulator fragments cloned between the \usename{35S} enhancer and minimal promoter. The position along the full-length insulator and the orientation (arrow pointing right, fwd; arrow pointing left, rev) of the fragments is indicated by arrows. Clusters of active fragments are shown as shaded areas. Insulators with highly orientation-dependent activity are circled.\nextentry
				\subfig{E} Correlation between insulator fragment enrichment and GC content for constructs with the \usename{35S} enhancer.\nextentry
				\subfig{F} Correlation between insulator fragment enrichment in \tobacco leaves in constructs with the indicated enhancers. The dashed line represents a y = x line fitted through the point corresponding to a control construct without an insulator (black dot).\nextentry
				Pearson's $R^2$, Spearman's $\rho$, and number ($n$) of constructs are indicated in \parensubfig{C}, \parensubfig{E}, and \parensubfig{F}. The dotted line in \parensubfig{D} and \parensubfig{E} represents the enrichment of a control construct without an insulator.
			}%
			\label{fig:fragments}%
		\end{fig}
		
		\begin{fig}
			\begin{tikzpicture}

	\pggfset{ylabel add to width = 3pt}
	
	%%% dual-luciferase construct
	\coordinate (DL construct) at (0, 0);
	
	\coordinate[shift = {(.2, -1.25)}] (construct) at (DL construct);
	
	\node[draw = black, thin, anchor = west, text depth = 0pt] (Luc) at ($(construct) + (.2, 0)$) {Luc};
	
	\draw[line width = .2cm, DarkOliveGreen3] (Luc.east) ++(.3, 0) coordinate (c1) -- ++(.5, 0) coordinate[pos = .5] (pro1) coordinate (c2);
	\draw[-{Stealth[round]}, thick] (c1) ++(-.5\pgflinewidth, 0) |- ++(-.35, .3);
	
	\node[draw = black, thin, anchor = west, text depth = 0pt] (BlpR) at ($(c2) + (.2, 0)$) {B/H\textsuperscript{R}};
	
	\draw[line width = .2cm, PaleGreen3] (BlpR.east) ++(.35, 0) coordinate (c3) -- ++(.5, 0) coordinate (c4);
	\draw[-{Stealth[round]}, thick] (c3) ++(-.5\pgflinewidth, 0) |- ++(-.35, .3);
	
	\draw[line width = .2cm, -Triangle Cap, 35Senhancer] (c4) ++(.4, 0) -- ++(.6, 0) coordinate[pos = .5] (enh) coordinate (c5);
	
	\fill[insulator] (c5) ++(.1, -.2) -- ++(60:.5) coordinate (ins) -- ++(-60:.5) -- cycle;
	
	\draw[line width = .2cm, 35Spromoter] (c5) ++(.7, 0) -- ++(.3, 0) coordinate[pos = .5] (pro2) coordinate (c6);
	\draw[-{Stealth[round]}, thick] (c6) ++(.5\pgflinewidth, 0) |- ++(.35, .3);
	
	\node[draw = black, thin, anchor = west, text depth = 0pt] (NLuc) at ($(c6) + (.3, 0)$) {NanoLuc};
	
	\coordinate[xshift = .2cm] (construct end) at (NLuc.east);
	
	\begin{pgfonlayer} {background}
		\draw[thick, {Bar[width = .1cm]}-] (construct) -- (Luc.west);
		\draw[thick] (Luc.east) -- (BlpR.west) (BlpR.east) -- (NLuc.west);
		\draw[thick, -{Bar[width = .1cm]}] (NLuc.east) -- (construct end);
	\end{pgfonlayer}
	
	\draw[Latex-] (ins) ++(0, .05) -- ++(0, .4) node[anchor = south, font = \bfseries] {insulator};
	\draw[Latex-] (pro1) ++(0, -.15) -- ++(0, -.4) node[anchor = north] {\textit{AtUBQ10} promoter};
	\draw[Latex-] (enh) ++(0, -.15) -- ++(-.3, -.4) node[anchor = north, align = center] {enhancer:\\\textcolor{35S}{\bfseries\usename{35S}} or \textcolor{AB80}{\bfseries\usename{AB80}}};
	\draw[Latex-] (pro2) ++(0, -.15) -- ++(.2, -.4) node[anchor = north, align = center] (l35Spr) {35S minimal\\promoter};

	\draw[decorate, decoration = {brace, raise = -.15cm, amplitude = .4cm, aspect = .825}] (l35Spr.south -| construct end) -- (l35Spr.south -| construct) coordinate[pos = .825, yshift = -.25cm] (curly);
	
	\draw[ultra thick, -Latex] (curly) ++(0, .75\pgflinewidth) -- ++(0, -.3) arc[start angle = 180, end angle = 270, radius = .8cm] -- ++(.3, 0) coordinate (plants);
	\planticon[scale = .55]{rice} at ($(plants) + (.7, .4)$);
	\planticon[scale = .55]{arabidopsis} at ($(plants) + (.9, -.4)$);
	\node [anchor = north east, align = center, inner xsep = 0pt] at (curly) {integrate\\into\\\rice or\\\Arabidopsis\\genome};
	
	\draw[ultra thick, -Latex] (plants) ++(1.4, 0) -- ++(.8, 0) node[anchor = west, align = center] {measure\\luciferase (Luc)\\and nanoluciferase\\(NanoLuc) activity};
	
	
	%%% Arabidopsis DL FLins: boxplots
	\coordinate[xshift = \columnsep] (FLins Arabidopsis) at (DL construct -| construct end);
	
	\distance{FLins Arabidopsis}{\textwidth, 0}
	
	\begin{pggfplot}[%
		at = {(FLins Arabidopsis)},
		total width = \xdistance,
		ylabel = \dlratio,
		xticklabels are csnames,
		xticklabel style = {font = \vphantom{\usename{lambda-EXOB}}, rotate = 45, anchor = north east},
		row title = Arabidopsis,
		title is csname,
		title color from name,
		enlarge y limits = {lower, value = 0.1},
		tight y,
		ylabel add to width = .5\baselineskip,
	]{DL_FLins_Arabidopsis}
		
		\pggf_annotate(zeroline)
		
		\pggf_hline()
	
		\pggf_boxplot(
			style from column = {fill = xticklabels},
			outlier options = {opacity = .5},
			outlier style from column = {xticklabels},
		)
	
	\end{pggfplot}
	
	
	%%% Arabidopsis DL FLins: correlation
	\coordinate[yshift = -.5\columnsep] (cor FLins Arabidopsis) at (DL construct |- pggfplot c1r1.below south);
	
	\begin{pggfplot}[%
		at = {(cor FLins Arabidopsis)},
		total width = \twocolumnwidth,
		xlabel = \textbf{Plant STARR-seq:} \enrichment,
		ylabel = \textbf{stable lines:} \dlratio,
		row title = Arabidopsis/tobacco,
		title is csname,
		col title color from name,
		row title style from name,
		row title style = {shading angle = 0},
		legend columns = 2,
		legend style = {
			at = {(1, 0)},
			anchor = south east,
		},
		legend cell align = right,
		legend plot pos = right,
		scatter styles = {empty legend, noEnh, empty legend, noIns, empty legend, lambda-EXOB, empty legend, BEAD-1C, UASrpg, sIns1, sIns2, gypsy}
	]{DL_cor_Flins_Arabidopsis}
	
		\pggf_trendline()
	
		\pggf_scatter(
			mark size = 2,
			style from column = insulator,
			y error = CI
		)
	
		\pggf_annotate(
			scatter legend*,
			only facets = 2
		)
	
		\pggf_stats(stats = {n,spearman,rsquare})
	
	\end{pggfplot}
	
	
	%%% rice DL FLins: boxplots
	\coordinate[xshift = \columnsep] (FLins rice) at (pggfplot c2r1.outer north east);
	
	\begin{pggfplot}[%
		at = {(FLins rice)},
		total width = .9\fourcolumnwidth,
		ylabel = \dlratio,
		xticklabels are csnames,
		xticklabel style = {font = \vphantom{\usename{sIns1}}, rotate = 45, anchor = north east},
		col title = 35S,
		row title = rice,
		title is csname,
		title color from name,
		enlarge y limits = {lower, value = 0.1},
		tight y
	]{DL_FLins_rice}
		
		\pggf_annotate(zeroline)
		
		\pggf_hline()
	
		\pggf_boxplot(
			style from column = {fill = xticklabels},
			outlier options = {opacity = .5},
			outlier style from column = {xticklabels},
		)
	
	\end{pggfplot}
	
	
	%%% rice DL FLins: correlation
	\coordinate[xshift = \columnsep] (cor FLins rice) at (pggfplot c1r1.outer north east);
	
	\begin{pggfplot}[%
		at = {(cor FLins rice)},
		total width = 1.1\fourcolumnwidth,
		xlabel = \textbf{Plant STARR-seq:} \enrichment,
		ylabel = \textbf{stable lines:} \dlratio,
		xlabel style = {xshift = -1.5pt},
		col title = 35S,
		row title = rice/maize,
		title is csname,
		col title color from name,
		row title style from name,
		row title style = {shading angle = 0},
		legend style = {
			at = {(1, 0)},
			anchor = south east,
		},
		legend cell align = right,
		legend plot pos = right,
		scatter styles = {noEnh, noIns, sIns1, sIns2}
	]{DL_cor_Flins_rice}
	
		\pggf_trendline()
	
		\pggf_scatter(
			mark size = 2,
			style from column = insulator,
			y error = CI
		)
	
		\pggf_annotate(scatter legend*)
	
		\pggf_stats(stats = {n,spearman,rsquare})
	
	\end{pggfplot}
	
	
	%%% Arabidopsis DL frags: boxplots
	\coordinate[yshift = -\columnsep] (frags Arabidopsis) at (DL construct |- pggfplot c1r1.below south);
	
	\begin{pggfplot}[%
		at = {(frags Arabidopsis)},
		total width = 1.1\threecolumnwidth,
		ylabel = \dlratio,
		xticklabels are csnames,
		xticklabel style = {font = \vphantom{\usename{lambda-EXOB}}, rotate = 45, anchor = north east, align = right},
		col title = 35S,
		row title = Arabidopsis,
		title is csname,
		title color from name,
		enlarge y limits = {lower, value = 0.1},
		tight y,
	]{DL_frags_Arabidopsis}
	
		\pggf_annotate(zeroline)
		
		\pggf_hline()
	
		\pggf_boxplot(
			style from column = {fill = xticklabels},
			outlier options = {opacity = .5},
			outlier style from column = {xticklabels},
		)
	
	\end{pggfplot}
	
	
	%%% Arabidopsis DL frags: correlation
	\coordinate[xshift = \columnsep] (cor frags Arabidopsis) at (pggfplot c1r1.outer north east);
	
	\begin{pggfplot}[%
		at = {(cor frags Arabidopsis)},
		total width = .9\threecolumnwidth,
		xlabel = \textbf{Plant STARR-seq:} \enrichment,
		ylabel = \textbf{stable lines:} \dlratio,
		ylabel style = {xshift = -.5em},
		col title = 35S,
		row title = Arabidopsis/tobacco,
		title is csname,
		col title color from name,
		row title style from name,
		enlarge y limits = {upper, value = .1},
		legend columns = 2,
		legend style = {
			at = {(0, 1)},
			anchor = north west,
			cells = {align = left}
		},
		legend cell align = left,
		legend plot pos = left,
		scatter styles = {noEnh,noIns,beta-phaseolin_1395_1564,TBS_756_925,lambda-EXOB_1_170,BEAD-1C_246_415,UASrpg_1_170,empty legend,gypsy_54_223,empty legend}
	]{DL_cor_frags_Arabidopsis}
	
		\pggf_trendline()
	
		\pggf_scatter(
			mark size = 2,
			style from column = insulator,
			y error = CI
		)
		
		\pggf_annotate(scatter legend*)
	
		\pggf_stats(stats = {n,spearman,rsquare}, position = south east)
	
	\end{pggfplot}
	
	
	%%% ELISA construct
	\coordinate[xshift = \columnsep] (ELISA construct) at (pggfplot c1r1.outer north east);
	
	\coordinate[shift = {(.2, -1.25)}] (construct) at (ELISA construct);
	\coordinate[xshift = .5cm] (c1) at (construct);
	
	\draw[line width = .2cm, -Triangle Cap, 35Senhancer] (c1) ++(.2, 0) -- ++(.6, 0) coordinate[pos = .5] (enh) coordinate (c2);
	
	\fill[insulator] (c2) ++(.1, -.2) -- ++(60:.5) coordinate (ins) -- ++(-60:.5) -- cycle;
	
	\draw[line width = .2cm, OliveDrab3] (c2) ++(.7, 0) -- ++(.5, 0) coordinate[pos = .5] (pro) coordinate (c3);
	\draw[-{Stealth[round]}, thick] (c3) ++(.5\pgflinewidth, 0) |- ++(.35, .3);
	
	\node[draw = black, thin, anchor = west, text depth = 0pt] (rep) at ($(c3) + (.3, 0)$) {reporter};
	
	\coordinate[xshift = .2cm] (c4) at (rep.east);
	\coordinate[xshift = .5cm] (construct end) at (c4);
	
	\begin{pgfonlayer} {background}
		\draw[thick, dashed] (construct) -- (c1);
		\draw[thick] (c1) -- (rep.west) (rep.east) -- (c4);
		\draw[thick, dashed] (c4) -- (construct end);
	\end{pgfonlayer}
	
	\draw[Latex-] (ins) ++(0, .05) -- ++(0, .4) node[anchor = south, font = \bfseries] {insulator};
	\draw[Latex-] (enh) ++(0, -.15) -- ++(0, -.4) node[anchor = north] {35S enhancer};
	\draw[Latex-] (pro) ++(0, -.15) -- ++(.3, -.4) node[anchor = north] (lpro) {ZmPro};
	
	\draw[decorate, decoration = {brace, raise = -.15cm, amplitude = .4cm, aspect = .9}] (lpro.south -| construct end) -- (lpro.south -| construct) coordinate[pos = .9, yshift = -.25cm] (curly);
	
	\draw[ultra thick, -Latex] (curly) ++(0, .75\pgflinewidth) -- ++(0, -.4) arc[start angle = 180, end angle = 270, radius = .8cm] -- ++(.5, 0) coordinate[yshift = .1cm] (maize);
	\planticon[scale = .45]{maize} at ($(maize) + (.45, -.15)$);
	\node [anchor = north west, align = center, shift = {(.1, .15)}] at (curly) {site-directed\\integration into\\\maize genome};
	
	\draw[ultra thick, -Latex] (maize) ++(1, -.1) -- ++(.8, 0) node[anchor = west, align = center] {measure\\reporter gene\\expression with\\ELISA};

	
	%%% maize ELISA frags: boxplots
	\coordinate[yshift = -1.5\columnsep] (frags maize) at (DL construct |- pggfplot c1r1.below south);
	
	\begin{pggfplot}[%
		at = {(frags maize)},
		total width = 1.1\threecolumnwidth,
		ylabel = ELISA (ppm/TP),
		xticklabels are csnames,
		xticklabel style = {font = \vphantom{\usename{lambda-EXOB}}, rotate = 45, anchor = north east, align = right},
		col title = 35S,
		row title = maize (R1 leaf),
		col title is csname,
		col title color from name,
		row title color = maize,
		enlarge y limits = {lower, value = 0.1},
		tight y,
	]{ELISA_R1_leaf}
	
		\pggf_annotate(zeroline)
		
		\pggf_hline()
	
		\pggf_boxplot(
			style from column = {fill = xticklabels},
		)
	
	\end{pggfplot}
	
	
	%%% maize ELISA frags: correlation
	\coordinate[xshift = \columnsep] (cor frags maize) at (pggfplot c1r1.outer north east);
	
	\begin{pggfplot}[%
		at = {(cor frags maize)},
		total width = .9\threecolumnwidth,
		xlabel = \textbf{Plant STARR-seq:} \enrichment,
		ylabel = \textbf{stable lines:} ELISA (ppm/TP),
		ylabel style = {xshift = -.5em},
		col title = 35S,
		row title = maize/maize,
		title is csname,
		col title color from name,
		row title style from name,
		enlarge y limits = {lower, value = .3},
		legend columns = 2,
		legend style = {
			at = {(1, 0)},
			anchor = south east,
			cells = {align = left}
		},
		legend cell align = right,
		legend plot pos = right,
		scatter styles = {empty legend,noEnh,empty legend,noIns,beta-phaseolin_781_612,beta-phaseolin_1395_1564,beta-phaseolin_2356_2187,TBS_588_757,lambda-EXOB_666_497,sIns2_335_504}
	]{ELISA_cor_R1_leaf}
	
		\pggf_trendline()
	
		\pggf_scatter(
			mark size = 2,
			style from column = id,
			y error = CI
		)
		
		\pggf_annotate(scatter legend*)
	
		\pggf_stats(stats = {n,spearman,rsquare})
	
	\end{pggfplot}
	
	
	%%% maize ELISA frags: correlation across tissues
	\coordinate[xshift = \columnsep] (tissues) at (pggfplot c1r1.outer north east);
	
	\distance{pggfplot c1r1.south}{pggfplot c1r1.above north}
	
	\begin{pggfplot}[%
		at = {(pggfplot c1r1.above north -| \textwidth, 0)},
		anchor = north east,
		total width = \threecolumnwidth,
		height = \ydistance,
		ylabel add to width = 32pt,
		ticklabels are csnames,
		xticklabel style = {font = \vphantom{\usename{lambda-EXOB}}, rotate = 45, anchor = north east, align = right},
		colormap={whiteblue}{color(0cm)=(white); color(1cm)=(RoyalBlue1)},
		facet 1/.append style = {
			pggf colorbar = {%
				title = {correlation ($R^2$)},
				/pgf/number format/fixed zerofill,
				/pgf/number format/precision = 1,
				at = {(1, .975)},
				anchor = north east,
			}
		},
		point meta min = 0,
		point meta max = 1,
		axis x line = bottom,
		axis y line = left,
		tight y
	]{ELISA_tissues}
	
		\pggf_heatmap(
			border,
			nodes near coords = {\pgfmathfloatifflags{\pgfplotspointmeta}{3}{}{\pgfmathprintnumber[fixed, fixed zerofill, precision = 2]{\pgfplotspointmeta}}},
			nodes near coords style = {anchor = center, font = \figsmall}
		)
	
	\end{pggfplot}

	
	%%% subfigure labels
	\subfiglabel{DL construct}
	\subfiglabel{FLins Arabidopsis}
	\subfiglabel{cor FLins Arabidopsis}
	\subfiglabel{FLins rice}
	\subfiglabel{cor FLins rice}
	\subfiglabel{frags Arabidopsis}
	\subfiglabel{cor frags Arabidopsis}
	\subfiglabel{ELISA construct}
	\subfiglabel{frags maize}
	\subfiglabel{cor frags maize}
	\subfiglabel{tissues}
	
\end{tikzpicture}%
			\nextpagecaption{%
				\captiontitle{Insulators are active in stable transgenic lines in \Arabidopsis, \rice, and \maize}%
				\subfig{A} Transgenic \Arabidopsis and \rice lines were generated with T-DNAs harboring a constitutively expressed luciferase (Luc) gene and a nanoluciferase (NanoLuc) gene under control of a 35S minimal promoter coupled to the \usename{35S} or \usename{AB80} enhancer (as indicated above the plots) with insulator candidates inserted between the enhancer and promoter. Nanoluciferase activity was measured in at least 4 plants from these lines and normalized to the activity of luciferase. The NanoLuc/Luc ratio was normalized to a control construct without an enhancer or insulator (noEnh; log2 set to 0).\nextentry
				\subfigtwo{B}{C} The activity of full-length insulators was measured in \Arabidopsis lines \parensubfig{B} and compared to the corresponding results from Plant STARR-seq in \tobacco leaves \parensubfig{C}.\nextentry
				\subfigtwo{D}{E} The activity of synthetic full-length insulators was measured in \rice lines \parensubfig{D} and compared to the corresponding results from Plant STARR-seq in \maize protoplasts \parensubfig{E}.\nextentry
				\subfigtwo{F}{G} The activity of insulator fragments was measured in \Arabidopsis lines \parensubfig{F} and compared to the corresponding results from Plant STARR-seq in \tobacco leaves \parensubfig{G}.\nextentry
				\subfig{H} For transgenic \maize lines, a reporter gene driven by the constitutive, moderate-strength \textit{ZmGOS2} promoter and an upstream \usename{35S} enhancer was created and insulator fragments were inserted between the enhancer and promoter. The reporter gene cassette was inserted in the maize genome by site-directed integration and the expression of the reporter gene was measured in various tissues/developmental stages by ELISA.\nextentry
				\subfigtwo{I}{J} The activity of insulator fragments was measured in R1 leaves of transgenic \maize lines \parensubfig{I} and compared to the corresponding results from Plant STARR-seq in \maize protoplasts \parensubfig{J}.\nextentry
				\subfig{K} Correlation (Pearson's $R^2$) between the expression of all tested constructs across different tissues and developmental stages. The correlation with Plant STARR-seq results from \maize protoplasts is also shown.\nextentry
				Box plots in \parensubfig{B}, \parensubfig{D}, \parensubfig{F}, and \parensubfig{I} represent the median (center line), upper and lower quartiles (box limits), 1.5$\times$ interquartile range (whiskers), and outliers (points) for all corresponding samples from two to three independent replicates. Numbers at the bottom of each box plot indicate the number of samples in each group. For groups with less than 10 samples, individual data points are shown as black dots. In \parensubfig{C}, \parensubfig{E}, \parensubfig{G}, and \parensubfig{J}, the dashed line represents a linear regression line and error bars represent the 95\% confidence interval. Pearson's $R^2$, Spearman's $\rho$, and number ($n$) of constructs are indicated. The dotted line in \parensubfig{B}, \parensubfig{D}, \parensubfig{F} and \parensubfig{I} represents the median enrichment of a control construct without an insulator, and the dashed line in \parensubfig{I} represents the median enrichment of a control construct without an insulator and without the 35S enhancer. 
			}%
			\label{fig:insDL}
		\end{fig}
		
		\begin{fig}
			\begin{tikzpicture}

	%%% constructs
	\coordinate (constructs) at (0, 0);
	
	% single insulator
	\node[anchor = west, text depth = 0pt, node font = \figlarge\bfseries, shift = {(\columnsep, -.5cm)}] (c1l) at (constructs) {single};
		
	\coordinate (construct 1) at (c1l.east);
	
	\draw[line width = .2cm, -Triangle Cap, 35Senhancer]  ($(construct 1) + (.2, 0)$) -- ++(.6, 0) coordinate[pos = .5] (enh);

	\fill[insulator] (construct 1) ++(.9, -.2) -- ++(60:.5) coordinate (ins) -- ++(-60:.5) -- cycle;

	\draw[line width = .2cm, 35Spromoter] (construct 1) ++(1.5, 0) -- ++(.3, 0) coordinate[pos = .5] (pro);
	\draw[-{Stealth[round]}, semithick] (pro) ++(.15cm + .5\pgflinewidth, 0) |- ++(.35, .3);
	
	\node[draw = black, thin, anchor = west, minimum width = 1cm, outer xsep = 0pt] (GFP) at ($(pro) + (.45, 0)$) {GFP};
	
	\begin{pgfonlayer} {background}
		\fill[Orchid1] (GFP.north west) ++(.15, 0) rectangle ($(GFP.south west) + (.05, 0)$);
	\end{pgfonlayer}
	
	\coordinate[xshift = .2cm] (construct 1 end) at (GFP.east);
	
	\begin{pgfonlayer} {background}
		\draw[thick, {Bar[width = .1cm]}-] (construct 1) -- (GFP.west);
		\draw[thick, -{Bar[width = .1cm]}] (GFP.east) -- (construct 1 end);
	\end{pgfonlayer}
	
	% double insulator
	\node[anchor = west, text depth = 0pt, node font = \figlarge\bfseries, xshift = 3\columnsep] (c2l) at (construct 1 end) {double};
	
	\coordinate (construct 2) at (c2l.east);
	
	\draw[line width = .2cm, -Triangle Cap, 35Senhancer]  ($(construct 2) + (.2, 0)$) -- ++(.6, 0) coordinate[pos = .5] (enh);

	\fill[insulator] (construct 2) ++(.9, -.2) -- ++(60:.5) coordinate (ins1) -- ++(-60:.5) -- cycle;
	\fill[insulator] (construct 2) ++(1.5, -.2) -- ++(60:.5) coordinate (ins2) -- ++(-60:.5) -- cycle;

	\draw[line width = .2cm, 35Spromoter] (construct 2) ++(2.1, 0) -- ++(.3, 0) coordinate[pos = .5] (pro);
	\draw[-{Stealth[round]}, semithick] (pro) ++(.15cm + .5\pgflinewidth, 0) |- ++(.35, .3);
	
	\node[draw = black, thin, anchor = west, minimum width = 1cm, outer xsep = 0pt] (GFP) at ($(pro) + (.45, 0)$) {GFP};
	
	\begin{pgfonlayer} {background}
		\fill[MediumPurple2] (GFP.north west) ++(.15, 0) rectangle ($(GFP.south west) + (.05, 0)$);
	\end{pgfonlayer}
	
	\coordinate[xshift = .2cm] (construct 2 end) at (GFP.east);
	
	\begin{pgfonlayer} {background}
		\draw[thick, {Bar[width = .1cm]}-] (construct 2) -- (GFP.west);
		\draw[thick, -{Bar[width = .1cm]}] (GFP.east) -- (construct 2 end);
	\end{pgfonlayer}
	
	% triple insulator
	\node[anchor = west, text depth = 0pt, node font = \figlarge\bfseries, xshift = 3\columnsep] (c3l) at (construct 2 end) {triple};
	
	\coordinate (construct 3) at (c3l.east);
	
	\draw[line width = .2cm, -Triangle Cap, 35Senhancer]  ($(construct 3) + (.2, 0)$) -- ++(.6, 0) coordinate[pos = .5] (enh);

	\fill[insulator!50] (construct 3) ++(.9, -.2) -- ++(60:.5) coordinate (ins1) -- ++(-60:.5) -- cycle;
	\fill[insulator] (construct 3) ++(1.5, -.2) -- ++(60:.5) coordinate (ins2) -- ++(-60:.5) -- cycle;
	\fill[insulator] (construct 3) ++(2.1, -.2) -- ++(60:.5) coordinate (ins3) -- ++(-60:.5) -- cycle;

	\draw[line width = .2cm, 35Spromoter] (construct 3) ++(2.7, 0) -- ++(.3, 0) coordinate[pos = .5] (pro);
	\draw[-{Stealth[round]}, semithick] (pro) ++(.15cm + .5\pgflinewidth, 0) |- ++(.35, .3);
	
	\node[draw = black, thin, anchor = west, minimum width = 1cm, outer xsep = 0pt] (GFP) at ($(pro) + (.45, 0)$) {GFP};
	
	\begin{pgfonlayer} {background}
		\fill[HotPink1] (GFP.north west) ++(.15, 0) rectangle ($(GFP.south west) + (.05, 0)$);
	\end{pgfonlayer}
	
	\coordinate[xshift = .2cm] (construct 3 end) at (GFP.east);
	
	\begin{pgfonlayer} {background}
		\draw[thick, {Bar[width = .1cm]}-] (construct 3) -- (GFP.west);
		\draw[thick, -{Bar[width = .1cm]}] (GFP.east) -- (construct 3 end);
	\end{pgfonlayer}
	
	
	% legend
	\coordinate[yshift = -1.2cm] (legend) at (constructs);
	
	\draw[line width = .2cm, -Triangle Cap, 35Senhancer]  ($(legend) + (.3, 0)$) -- ++(.6, 0);
	\node[anchor = west, text depth = 0pt, xshift = .9cm] (enh label) at (legend) {35S enhancer};
	
	\draw[line width = .2cm, 35Spromoter] (enh label.east) ++(\columnsep, 0) -- ++(.3, 0);
	\node[anchor = west, text depth = 0pt, xshift = \columnsep + .3cm] (pro label) at (enh label.east) {35S minimal promoter};
	
	\fill[insulator] (pro label.east) ++(\columnsep, -.2) -- ++(60:.5) coordinate (ins2) -- ++(-60:.5) -- cycle;
	\node[anchor = west, text depth = 0pt, xshift = \columnsep + .5cm] (ins label) at (pro label.east) {32 insulator fragments in either orientation};
	
	\fill[insulator!50] (ins label.east) ++(\columnsep, -.2) -- ++(60:.5) coordinate (ins2) -- ++(-60:.5) -- cycle;
	\node[anchor = west, text depth = 0pt, xshift = \columnsep + .5cm] (ins label 2) at (ins label.east) {5 strong insulator fragments in fixed orientation};
	
	
	%%% iFC enrichment
	\coordinate[yshift = -.2cm - \columnsep] (iFC) at (constructs |- legend);
	
	\begin{pggfplot}[%
		at = {(iFC)},
		total width = \threecolumnwidth,
		xlabel = number of insulator fragments,
		ylabel = \enrichment,
		enlarge y limits = {lower, value = 0.1},
		title is csname,
		title color from name
	]{iFC_enrichment}
	
		\pggf_annotate(zeroline)
	
		\pggf_hline()
	
		\pggf_violin(
			color from col name,
			shade = left
		)
	
	\end{pggfplot}
	
	
	%%% iFC prediction
	\coordinate (prediction) at (iFC -| \textwidth - \twothirdcolumnwidth, 0);
	
	\begin{pggfplot}[%
		at = {(prediction)},
		total width = \textwidth - \threecolumnwidth - \fourcolumnwidth - 2\columnsep,
		xlabel = {\textbf{prediction:} \enrichment},
		ylabel = {\textbf{measurement:} \enrichment},
		col title is csname,
		col title color from name,
		xytick = {-10, -8, ..., 10},
		facet 2/.append style = {pggf colorbar = {log2}},
	]{iFC_cor_prediction}
	
		\pggf_annotate(diagonal)
	
		\pggf_hexbin(bins = 50)
	
		\pggf_stats(stats = {n,spearman,rsquare})
	
	\end{pggfplot}
	
	
	%%% iFC model coefficients
	\coordinate (coefficients) at (iFC -| \textwidth - \fourcolumnwidth, 0);
	
	% construct
	\coordinate[xshift = -3.4cm] (construct) at (xlabel -| \textwidth, 0);
	
	\draw[line width = .2cm, -Triangle Cap, 35Senhancer]  (construct) ++(.2, 0) -- ++(.6, 0) coordinate[pos = .5] (enh);

	\fill[insulator!50] (construct) ++(.9, -.2) -- ++(60:.5) coordinate (ins1) -- ++(-60:.5) -- cycle;
	\fill[insulator] (construct) ++(1.5, -.2) -- ++(60:.5) coordinate (ins2) -- ++(-60:.5) -- cycle;
	\fill[insulator] (construct) ++(2.1, -.2) -- ++(60:.5) coordinate (ins3) -- ++(-60:.5) -- cycle;

	\draw[line width = .2cm, 35Spromoter] (construct) ++(2.7, 0) -- ++(.3, 0) coordinate[pos = .5] (pro);
	\draw[-{Stealth[round]}, semithick] (pro) ++(.15cm + .5\pgflinewidth, 0) |- ++(.35,.3);
	
	\begin{pgfonlayer} {background}
		\draw[thick, {Bar[width = .1cm]}-] (construct) -- ++(3, 0);
		\draw[thick, dashed] (construct) ++(3, 0) -- ++(.4, 0); 
	\end{pgfonlayer}
	
	% plot
	\begin{pggfplot}[%
		at = {(coefficients)},
		total width = \fourcolumnwidth,
		ylabel = coefficient,
		ylabel add to width = .5\baselineskip,
		xticklabel = \empty,
		col title = linear model,
		ymin = 0,
		legend style = {
			at = {(0, 1)},
			anchor = north west,
			cells = {align = left}
		},
		legend cell align = left,
		legend plot pos = left,
	]{iFC_model_coef}
	
		\pggf_bar(
			ybar,
			style from column = {fill = group}
		)
		
		\pggf_annotate(
			annotation = {
				\coordinate (c1) at (1, 0);
				\coordinate (c2) at (2, 0);
				\coordinate (c3) at (3, 0);
			}
		)
	
	\end{pggfplot}
	
	% connect construct to plot
	\begin{pgfonlayer} {background}
		\foreach \x in {1, ..., 3}{
			\draw[gray, line cap = round] (c\x) ++(0, -\groupplotsep) to[out = -90, in = 90] (ins\x);
		}
	\end{pgfonlayer}


	%%% stable maize lines: ELISA
	\coordinate[yshift = -\columnsep - .2cm] (ELISA) at (constructs |- construct);
	
	\begin{pggfplot}[%
		at = {(ELISA)},
		total width = 1.1\threecolumnwidth,
		ylabel = ELISA (ppm/TP),
		xticklabels are csnames,
		xticklabel style = {font = \vphantom{\usename{lambda-EXOB}}, rotate = 45, anchor = north east, align = right},
		col title = maize (R1 leaf),
		col title color = maize,
		enlarge y limits = {lower, value = 0.1},
		tight y,
	]{ELISA_stacked_R1_leaf}
	
		\pggf_annotate(zeroline)
		
		\pggf_hline(
			noEnh/.style = dashed,
			noIns/.style = {draw = black},
			style from column = intercept_id
		)
	
		\pggf_boxplot(
			style from column = {fill = xticklabels},
		)
	
	\end{pggfplot}
	
	
	%%% stable ELISA lines: correlation with Plant STARR-seq
	\coordinate[xshift = \columnsep] (ELISA cor) at (pggfplot c1r1.outer north east);
	
	\begin{pggfplot}[%
		at = {(ELISA cor)},
		total width = .9\threecolumnwidth,
		xlabel = \textbf{Plant STARR-seq:} \enrichment,
		ylabel = \textbf{stable lines:} ELISA (ppm/TP),
		ylabel style = {xshift = -.5em},
		col title = maize/maize,
		title is csname,
		col title style from name,
		legend columns = 3,
		legend style = {
			at = {(1, 0)},
			anchor = south east
		},
		legend cell align = right,
		legend plot pos = right,
		scatter styles = {noEnh,noIns,D2,T30,T21,T27,T32,T24,T25,T19,T9}
	]{ELISA_stacked_cor_R1_leaf}
	
		\pggf_trendline()
	
		\pggf_scatter(
			mark size = 2,
			style from column = short_id,
			y error = CI
		)
		
		\pggf_annotate(scatter legend*)
	
		\pggf_stats(
			stats = {n,spearman,rsquare},
			text mark style = {
				anchor = north east,
				at = (current axis.north east),
				xshift = -1.5\baselineskip
			}
		)
	
	\end{pggfplot}
	
	
	%%% stable maize lines: correlation
	\coordinate[xshift = \columnsep] (tissues) at (pggfplot c1r1.outer north east);
	
	\distance{pggfplot c1r1.south}{pggfplot c1r1.above north}
	
	\begin{pggfplot}[%
		at = {(pggfplot c1r1.above north -| \textwidth, 0)},
		anchor = north east,
		total width = \threecolumnwidth,
		height = \ydistance,
		ylabel add to width = 32pt,
		ticklabels are csnames,
		xticklabel style = {font = \vphantom{\usename{lambda-EXOB}}, rotate = 45, anchor = north east, align = right},
		colormap={whiteblue}{color(0cm)=(white); color(1cm)=(RoyalBlue1)},
		facet 1/.append style = {
			pggf colorbar = {%
				title = {correlation ($R^2$)},
				/pgf/number format/fixed zerofill,
				/pgf/number format/precision = 1,
				at = {(1, .975)},
				anchor = north east,
			}
		},
		point meta min = 0,
		point meta max = 1,
		axis x line = bottom,
		axis y line = left,
		tight y
	]{ELISA_stacked_tissues}
	
		\pggf_heatmap(
			border,
			nodes near coords = {\pgfmathfloatifflags{\pgfplotspointmeta}{3}{}{\pgfmathprintnumber[fixed, fixed zerofill, precision = 2]{\pgfplotspointmeta}}},
			nodes near coords style = {anchor = center, font = \figsmall}
		)
	
	\end{pggfplot}
	
	
	%%% subfigure labels
	\subfiglabel{constructs}
	\subfiglabel{iFC}
	\subfiglabel{prediction}
	\subfiglabel{coefficients}
	\subfiglabel{ELISA}
	\subfiglabel{ELISA cor}
	\subfiglabel{tissues}

\end{tikzpicture}%
			\caption{%
				\captiontitle{Insulator fragments can be stacked to create very strong enhancer-blocking insulators}%
				\subfig{A} One, two, or three 170-bp fragments of known insulators were cloned between a \usename{35S} enhancer and a 35S minimal promoter driving the expression of a barcoded GFP reporter gene.\nextentry
				\subfig{B} All insulator constructs were pooled and subjected to Plant STARR-seq in \tobacco leaves (\tobacco) and \maize protoplasts (\maize). Reporter mRNA enrichment was normalized to a control construct without an enhancer or insulator (log2 set to 0). Violin plots are as defined in \cref{fig:fragments}\subfigunformatted{B}.\nextentry
				\subfig{C} A linear model was trained to predict the enrichment of stacked insulator constructs based on the activity of individual insulator fragments and their position within the construct. The correlation between the model's prediction (prediction) and experimentally determined enrichment values (measurement) is shown as a hexbin plot (color represents the count of points in each hexagon). Pearson's $R^2$, Spearman's $\rho$, and number ($n$) of fragments are indicated.\nextentry
				\subfig{D} Coefficients assigned by the linear model to insulator fragments in the indicated positions of the stacked constructs.\nextentry
				\subfigtwo{E}{F} The activity of insulator fragment combinations in constructs as in \cref{fig:insDL}\subfigunformatted{h} was measured in R1 leaves of transgenic \maize lines \parensubfig{E} and compared to the corresponding results from Plant STARR-seq in \maize protoplasts \parensubfig{F}. Box plots are as defined in \cref{fig:insDL}. The enrichment of a control construct without an insulator (noIns) is indicated as a dotted line. In \parensubfig{F}, the dashed line represents a linear regression line and error bars represent the 95\% confidence interval. Pearson's $R^2$, Spearman's $\rho$, and number ($n$) of constructs are indicated.\nextentry
				\subfig{G} Correlation (Pearson's $R^2$) between the expression of all tested constructs across different tissues and developmental stages. The correlation with Plant STARR-seq results from \maize protoplasts is also shown.
				The dotted line in \parensubfig{B} and \parensubfig{E} represents the enrichment of a control construct without an insulator, and the dashed line in \parensubfig{E} represents the enrichment of a control construct without an insulator and without the 35S enhancer.
			}%
			\label{fig:stacking}%
		\end{fig}
		
		\begin{fig}
			\begin{tikzpicture}

	%%% constructs iDE library
	\coordinate (constructs iDE) at (0, 0);
	
	% with 35S enhancer
	\coordinate[shift = {(.05, -1.5)}] (construct) at (constructs iDE);
	
	\draw[line width = .2cm, -Triangle Cap, 35Senhancer]  ($(construct) + (.2, 0)$) -- ++(.6, 0) coordinate[pos = .5] (enh up);
	
	\draw[Latex-] (enh up) ++(0, .15) -- ++(0, .4) coordinate (lenh);
	\node[anchor = south west, text depth = 0pt] (lwith) at (lenh -| constructs iDE) {upstream enhancer: \textcolor{35S}{\bfseries\usename{35S}}};

	\fill[insulator] (construct) ++(.9, -.2) -- ++(60:.5) coordinate (ins) -- ++(-60:.5) -- cycle;
	
	\draw[line width = .2cm, -Triangle Cap, AB80]  ($(construct) + (1.5, 0)$) -- ++(.6, 0) coordinate[pos = .5] (enh down);
	
	\draw[Latex-] (enh down) ++(0, -.15) -- ++(0, -.4) node[anchor = north, align = center] (lenh down) {downstream enhancer:\\\textcolor{AB80}{\bfseries\usename{AB80}} or \textcolor{Cab-1}{\bfseries\usename{Cab-1}}};

	\draw[line width = .2cm, 35Spromoter] ($(construct) + (2.3, 0)$) -- ++(.3, 0) coordinate[pos = .5] (pro);
	\draw[-{Stealth[round]}, semithick] (pro) ++(.15cm + .5\pgflinewidth, 0) |- ++(.35, .3);
	
	\node[draw = black, thin, anchor = west, minimum width = 1cm, outer xsep = 0pt] (GFP) at ($(pro) + (.45, 0)$) {GFP};
	
	\begin{pgfonlayer} {background}
		\fill[Orchid1] (GFP.north west) ++(.15, 0) rectangle ($(GFP.south west) + (.05, 0)$);
	\end{pgfonlayer}
	
	\coordinate[xshift = .2cm] (construct end) at (GFP.east);
	
	\begin{pgfonlayer} {background}
		\draw[thick, {Bar[width = .1cm]}-] (construct) -- (GFP.west);
		\draw[thick, -{Bar[width = .1cm]}] (GFP.east) -- (construct end);
	\end{pgfonlayer}
	
	
	% without 35S enhancer
	\coordinate[yshift = -.55cm] (construct 2) at (construct |- lenh down.south);
	
	\draw[thick, densely dotted, gray, line cap = round] (construct 2) ++(.7, 0) coordinate (enh up) circle (.125cm);
	
	\draw[Latex-] (enh up) ++(0, -.15) -- ++(0, -.4) coordinate (lenh);
	\node[anchor = north west, text depth = 0pt, gray] (lwithout) at (lenh -| constructs iDE) {no upstream enhancer};

	\fill[insulator] (construct 2) ++(.9, -.2) -- ++(60:.5) coordinate (ins) -- ++(-60:.5) -- cycle;
	
	\draw[line width = .2cm, -Triangle Cap, AB80]  ($(construct 2) + (1.5, 0)$) -- ++(.6, 0) coordinate[pos = .5] (enh down);
	
	\draw[Latex-] (enh down) ++(0, .15) -- ++(0, .4);

	\draw[line width = .2cm, 35Spromoter] ($(construct 2) + (2.3, 0)$) -- ++(.3, 0) coordinate[pos = .5] (pro);
	\draw[-{Stealth[round]}, semithick] (pro) ++(.15cm + .5\pgflinewidth, 0) |- ++(.35, .3);
	
	\node[draw = black, thin, anchor = west, minimum width = 1cm, outer xsep = 0pt] (GFP) at ($(pro) + (.45, 0)$) {GFP};
	
	\begin{pgfonlayer} {background}
		\fill[MediumPurple2] (GFP.north west) ++(.15, 0) rectangle ($(GFP.south west) + (.05, 0)$);
	\end{pgfonlayer}
	
	\coordinate[xshift = .2cm] (construct 2 end) at (GFP.east);
	
	\begin{pgfonlayer} {background}
		\draw[thick, {Bar[width = .1cm]}-] (construct 2) ++(.5, 0) -- (GFP.west);
		\draw[thick, -{Bar[width = .1cm]}] (GFP.east) -- (construct 2 end);
	\end{pgfonlayer}
	
	% names
	\node[anchor = south, node font = \figlarge\bfseries, shift = {(.5\fourcolumnwidth, .25\columnsep)}] at (constructs iDE |- lwith.north) {\usename{with}};
	\node[anchor = north, node font = \figlarge\bfseries, shift = {(.5\fourcolumnwidth, -.25\columnsep)}] at (constructs iDE |- lwithout.south) {\usename{without}};
		
	
	%%% enrichment iDE
	\coordinate (enrichment iDE) at (constructs iDE -| \textwidth - \threequartercolumnwidth, 0);
	
	\begin{pggfplot}[%
		at = {(enrichment iDE)},
		total width = \threecolumnwidth,
		xlabel = downstream enhancer,
		ylabel = \enrichment,
		xticklabels are csnames,
		xticklabel style = {font = \vphantom{\usename{AB80}}},
		title is csname,
		title color from name,
		enlarge y limits = {lower, value = 0.1},
	]{iDE_enrichment}
	
		\pggf_annotate(zeroline)
	
		\pggf_violin(
			style from column = {fill* = xticklabels},
			extra mark = {mark size = 2.5}{noIns}
		)
	
	\end{pggfplot}
	
	
	%%% correlation +/- 35S enhancer
	\coordinate (cor 35S) at (constructs iDE -| \fourcolumnwidth + \threecolumnwidth + 2\columnsep, 0);
	
	\begin{pggfplot}[%
		at = {(cor 35S)},
		total width = {\textwidth - \fourcolumnwidth - \threecolumnwidth - 2\columnsep},
		xlabel = {\textbf{\usename{without}:} \enrichment},
		ylabel = {\textbf{\usename{with}:} \enrichment},
		title is csname,
		title color from name,
	]{iDE_cor_35S}
	
		\pggf_scatter(
			data = iDE_cor_35S_fragments,
			color from col name,
			opacity = .5
		)
		
		\pggf_scatter(
			data = iDE_cor_35S_control,
			mark size = 2.5
		)
		
		\pggf_stats(stats = {n,spearman,rsquare})
	
		\pggf_abline(gray, dashed)
	
	\end{pggfplot}
	
	
	%%% constructs i/sLib
	\coordinate[yshift = -\columnsep] (constructs i/sLib) at (constructs iDE |- xlabel.south);
	
	% insulator construct
	\coordinate[shift = {({.5 * (\fourcolumnwidth - 3.3cm)}, -1.25)}] (construct) at (constructs i/sLib);
	
	\draw[line width = .2cm, -Triangle Cap, 35Senhancer]  ($(construct) + (.2, 0)$) -- ++(.6, 0) coordinate[pos = .5] (enh);

	\fill[insulator] (construct) ++(.9, -.2) -- ++(60:.5) coordinate (ins) -- ++(-60:.5) -- cycle;
	
	\draw[Latex-] (construct -| ins) ++(0, -.25) -- ++(0, -.4) node[anchor = north, align = center, xshift = -.35cm] (lins/sil) {\textcolor{silencer}{\bfseries silencer}/\textcolor{insulator}{\bfseries insulator}\\candidates};

	\draw[line width = .2cm, 35Spromoter] (construct) ++(1.5, 0) -- ++(.3, 0) coordinate[pos = .5] (pro);
	\draw[-{Stealth[round]}, semithick] (pro) ++(.15cm + .5\pgflinewidth, 0) |- ++(.35,.3);
	
	\node[draw = black, thin, anchor = west, minimum width = 1cm, outer xsep = 0pt] (GFP) at ($(pro) + (.45, 0)$) {GFP};
	
	\begin{pgfonlayer} {background}
		\fill[Orchid1] (GFP.north west) ++(.15, 0) rectangle ($(GFP.south west) + (.05, 0)$);
	\end{pgfonlayer}
	
	\coordinate[xshift = .2cm] (construct end) at (GFP.east);
	
	\begin{pgfonlayer} {background}
		\draw[thick, {Bar[width = .1cm]}-] (construct) -- (GFP.west);
		\draw[thick, -{Bar[width = .1cm]}] (GFP.east) -- (construct end);
	\end{pgfonlayer}
	
	
	% silencer construct
	\coordinate[yshift = -.7cm] (construct 2) at (construct |- lins/sil.south);

	\fill[insulator] (construct 2) ++(.2, -.2) -- ++(60:.5) coordinate (ins) -- ++(-60:.5) -- cycle;
	
	\draw[Latex-] (ins) ++(0, .05) -- (lins/sil.south -| ins);
	
	\draw[line width = .2cm, -Triangle Cap, 35Senhancer]  ($(construct 2) + (.8, 0)$) -- ++(.6, 0) coordinate[pos = .5] (enh);

	\draw[line width = .2cm, 35Spromoter] (construct 2) ++(1.5, 0) -- ++(.3, 0) coordinate[pos = .5] (pro);
	\draw[-{Stealth[round]}, semithick] (pro) ++(.15cm + .5\pgflinewidth, 0) |- ++(.35,.3);
	
	\node[draw = black, thin, anchor = west, minimum width = 1cm, outer xsep = 0pt] (GFP) at ($(pro) + (.45, 0)$) {GFP};
	
	\begin{pgfonlayer} {background}
		\fill[MediumPurple2] (GFP.north west) ++(.15, 0) rectangle ($(GFP.south west) + (.05, 0)$);
	\end{pgfonlayer}
	
	\coordinate[xshift = .2cm] (construct 2 end) at (GFP.east);
	
	\begin{pgfonlayer} {background}
		\draw[thick, {Bar[width = .1cm]}-] (construct 2) -- (GFP.west);
		\draw[thick, -{Bar[width = .1cm]}] (GFP.east) -- (construct 2 end);
	\end{pgfonlayer}
	
	% names
	\node[anchor = south, node font = \figlarge\bfseries, shift = {(.5\fourcolumnwidth, \columnsep)}] at (constructs i/sLib |- construct) {insulator construct};
	\node[anchor = north, node font = \figlarge\bfseries, shift = {(.5\fourcolumnwidth, -\columnsep)}] at (constructs i/sLib |- construct 2) {silencer construct};
	
	
	%%% enrichment i/sLib
	\coordinate (enrichment i/sLib) at (constructs i/sLib -| \textwidth - \threequartercolumnwidth, 0);
	
	\begin{pggfplot}[%
		at = {(enrichment i/sLib)},
		total width = \threecolumnwidth,
		xlabel = construct,
		ylabel = \enrichment,
		xticklabels are csnames,
		xticklabel style = {font = \vphantom{\usename{lt}}},
		title is csname,
		title color from name,
		enlarge y limits = {lower, value = 0.1},
	]{isLib_enrichment}
	
		\pggf_annotate(zeroline)
		
		\pggf_hline()
	
		\pggf_violin(style from column = {fill* = xticklabels})
	
	\end{pggfplot}
	
	
	%%% insulator vs. silencer activity
	\coordinate (ins v sil) at (constructs i/sLib -| \fourcolumnwidth + \threecolumnwidth + 2\columnsep, 0);
	
	\begin{pggfplot}[%
		at = {(ins v sil)},
		total width = {\textwidth - \fourcolumnwidth - \threecolumnwidth - 2\columnsep},
		xlabel = {\textbf{\usename{insulator}:} \enrichment},
		ylabel = {\textbf{\usename{silencer}:} \enrichment},
		title is csname,
		title color from name,
	]{isLib_IvS}
	
		\pggf_annotate(diagonal)
	
		\pggf_trendline(
			thick,
			solid
		)
	
		\pggf_stats(
			stats = {slope,goodness of fit},
			position = south east
		)
	
		\pggf_scatter(
			color from col name,
			opacity = .5
		)
	
	\end{pggfplot}
	
	
	%%% subfigure labels
	\subfiglabel{constructs iDE}
	\subfiglabel{enrichment iDE}
	\subfiglabel{cor 35S}
	\subfiglabel{constructs i/sLib}
	\subfiglabel{enrichment i/sLib}
	\subfiglabel{ins v sil}
	
\end{tikzpicture}%
			\caption{%
				\captiontitle{Insulators exhibit silencer activity in some contexts}%
				\subfig{A} Insulator fragments (yellow triangle) were cloned upstream of a \usename{AB80} or \usename{Cab-1} enhancer and a 35S minimal promoter (green rectangle) driving the expression of a barcoded GFP reporter gene. Half of the constructs also harbored a \usename{35S} enhancer upstream of the insulator fragments (\usename{with}) while the other half lacked an upstream enhancer (\usename{without}).\nextentry
				\subfig{B} All constructs were pooled and subjected to Plant STARR-seq in \tobacco leaves. Reporter mRNA enrichment was normalized to a control construct without an enhancer or insulator (noEnh; log2 set to 0). The enrichment of a control construct without an insulator is indicated as a black dot.\nextentry
				\subfig{C}  Correlation between insulator fragment activity in constructs with or without the upstream \usename{35S} enhancer. The dashed line represents a y = x line fitted through the point corresponding to a control construct without an insulator (black dot).\nextentry
				\subfig{D} Insulator fragments (yellow triangle) were cloned in between (insulator construct) or upstream of (silencer construct) a \usename{35S} enhancer (blue arrow) and a 35S minimal promoter (green rectangle) driving the expression of a barcoded GFP reporter gene.\nextentry
				\subfig{E} All constructs were pooled and subjected to Plant STARR-seq in \tobacco leaves (\tobacco) or \maize protoplasts (\maize). Reporter mRNA enrichment was normalized to a control construct without an enhancer or insulator (noEnh; log2 set to 0). The enrichment of a control construct without an insulator is indicated as a dotted line.\nextentry
				\subfig{F} Comparison of the enrichment of insulator fragments in insulator or silencer constructs. A linear regression line is shown as a solid line and its slope and goodness-of-fit ($R^2$) is indicated.\nextentry
				Violin plots in \parensubfig{B} and \parensubfig{E} are as defined in \cref{fig:fragments}\subfigunformatted{B}.
			}%
			\label{fig:silencer}%
		\end{fig}
		
		\begin{fig}
			\begin{tikzpicture}

	%%% constructs
	\coordinate (constructs) at (0, 0);
	
	% insulator construct
	\coordinate[shift = {({.5 * (\fourcolumnwidth - 3.3cm)}, -1.5)}] (construct) at (constructs);
	
	\draw[line width = .2cm, -Triangle Cap, 35Senhancer]  ($(construct) + (.2, 0)$) -- ++(.6, 0) coordinate[pos = .5] (enh);

	\draw[Latex-] (enh) ++(0, -.15) -- ++(0, -.4) node[anchor = north, align = center, xshift = .35cm] (lenh) {8 different enhancers};

	\fill[insulator] (construct) ++(.9, -.2) -- ++(60:.5) coordinate (ins) -- ++(-60:.5) -- cycle;

	\draw[line width = .2cm, 35Spromoter] (construct) ++(1.5, 0) -- ++(.3, 0) coordinate[pos = .5] (pro);
	\draw[-{Stealth[round]}, semithick] (pro) ++(.15cm + .5\pgflinewidth, 0) |- ++(.35,.3);
	
	\node[draw = black, thin, anchor = west, minimum width = 1cm, outer xsep = 0pt] (GFP) at ($(pro) + (.45, 0)$) {GFP};
	
	\begin{pgfonlayer} {background}
		\fill[Orchid1] (GFP.north west) ++(.15, 0) rectangle ($(GFP.south west) + (.05, 0)$);
	\end{pgfonlayer}
	
	\coordinate[xshift = .2cm] (construct end) at (GFP.east);
	
	\begin{pgfonlayer} {background}
		\draw[thick, {Bar[width = .1cm]}-] (construct) -- (GFP.west);
		\draw[thick, -{Bar[width = .1cm]}] (GFP.east) -- (construct end);
	\end{pgfonlayer}
	
	
	% silencer construct
	\coordinate[yshift = -.55cm] (construct 2) at (construct |- lenh.south);

	\fill[insulator] (construct 2) ++(.2, -.2) -- ++(60:.5) coordinate (ins) -- ++(-60:.5) -- cycle;
	
	\draw[line width = .2cm, -Triangle Cap, 35Senhancer]  ($(construct 2) + (.8, 0)$) -- ++(.6, 0) coordinate[pos = .5] (enh);
	
	\draw[Latex-] (enh) ++(0, .15) -- (lenh.south -| enh);

	\draw[line width = .2cm, 35Spromoter] (construct 2) ++(1.5, 0) -- ++(.3, 0) coordinate[pos = .5] (pro);
	\draw[-{Stealth[round]}, semithick] (pro) ++(.15cm + .5\pgflinewidth, 0) |- ++(.35,.3);
	
	\node[draw = black, thin, anchor = west, minimum width = 1cm, outer xsep = 0pt] (GFP) at ($(pro) + (.45, 0)$) {GFP};
	
	\begin{pgfonlayer} {background}
		\fill[MediumPurple2] (GFP.north west) ++(.15, 0) rectangle ($(GFP.south west) + (.05, 0)$);
	\end{pgfonlayer}
	
	\coordinate[xshift = .2cm] (construct 2 end) at (GFP.east);
	
	\begin{pgfonlayer} {background}
		\draw[thick, {Bar[width = .1cm]}-] (construct 2) -- (GFP.west);
		\draw[thick, -{Bar[width = .1cm]}] (GFP.east) -- (construct 2 end);
	\end{pgfonlayer}
	
	% names
	\node[anchor = south, node font = \figlarge\bfseries, shift = {(.5\fourcolumnwidth, \columnsep)}] at (constructs |- construct) {insulator construct};
	\node[anchor = north, node font = \figlarge\bfseries, shift = {(.5\fourcolumnwidth, -\columnsep)}] at (constructs |- construct 2) {silencer construct};
	
	
	%%% enhancers
	\coordinate (enhancers) at (constructs -| \textwidth - \threequartercolumnwidth, 0);
	
	% redefine AB80/Cab-1 colors and define names/colors for additional enhancers
	\newcommand{\defenhancer}[2]{%
		\expandafter\def\csname#1\endcsname{#1\xspace}%
		\colorlet{#1}{#2}%
	}
	
	\defenhancer{Sl-774}{none!82.5!35S}
	\defenhancer{Zm-23177}{none!75!35S}
	\defenhancer{Sb-11289}{none!62.5!35S}
	\colorlet{AB80}{none!50!35S}
	\colorlet{Cab-1}{none!37.5!35S}
	\defenhancer{Sl-12881}{none!25!35S}
	\defenhancer{At-9661}{none!12.5!35S}
	
	% plot
	\begin{pggfplot}[%
		at = {(enhancers)},
		total width = \textwidth - \fourcolumnwidth - \threecolumnwidth - 2\columnsep,
		ylabel = \enrichment,
		xticklabels are csnames,
		xticklabel style = {font = \vphantom{\usename{35S}}, rotate = 45, anchor = north east},
		tight y,
		title is csname,
		title color from name,
		enlarge y limits = {lower, value = 0.1},
	]{ExI_enhancers}
	
		\pggf_annotate(zeroline)
	
		\pggf_boxplot(
			style from column = {fill* = xticklabels},
			outlier options = {opacity = .5},
			outlier style from column = {xticklabels}
		)
	
	\end{pggfplot}
	
	
	%%% insulator vs silencer
	\coordinate[yshift = -\columnsep] (IvS) at (constructs |- xlabel.north);
	
	\begin{pggfplot}[%
		at = {(IvS)},
		total width = \textwidth,
		xlabel = {\textbf{\usename{insulator}:} \enrichment},
		ylabel = {\textbf{\usename{silencer}:} \enrichment},
		title is csname,
		title color from name,
	]{ExI_IvS}
	
		\pggf_annotate(diagonal)
	
		\pggf_trendline(
			thick,
			solid
		)
	
		\pggf_stats(
			stats = {slope,goodness of fit},
			position = south east
		)
	
		\pggf_scatter(
			mark size = 1.5,
			color from col name
		)
	
	\end{pggfplot}
	
	
	%%% correlation between slope and enhancer strength
	\coordinate (cor slope) at (constructs -| \textwidth - \threecolumnwidth, 0);
	
	\begin{pggfplot}[%
		at = {(cor slope)},
		total width = \threecolumnwidth,
		xlabel = slope of regression line,
		ylabel = \enrichment,
		title is csname,
		title color from name,
	]{ExI_cor_slope}
	
		\pggf_trendline()
	
		\pggf_stats(
			stats = {n,spearman,rsquare},
			position = south west
		)
	
		\pggf_scatter(
			mark size = 1.5,
			color from col name
		)
	
	\end{pggfplot}
	

	%%% subfigure labels
	\subfiglabel{constructs}
	\subfiglabel{enhancers}
	\subfiglabel{IvS}
	\subfiglabel{cor slope}

\end{tikzpicture}%
			\caption{%
				\captiontitle{Silencer activity depends on enhancer strength}%
				\subfig{A} Selected insulators and insulator fragments (yellow triangle) were cloned in between (insulator construct) or upstream of (silencer construct) an enhancer and a 35S minimal promoter (green rectangle) driving the expression of a barcoded GFP reporter gene. Eight different enhancers were used to build these constructs. All constructs were pooled and subjected to Plant STARR-seq in \tobacco leaves (\tobacco) or \maize protoplasts (\maize).\nextentry
				\subfig{B} Strength of the eight enhancers in constructs without an insulator. Reporter mRNA enrichment was normalized to a control construct without an enhancer (none; log2 set to 0). Box plots represent the median (center line), upper and lower quartiles, and 1.5$\times$ interquartile range (whiskers) for all corresponding barcodes from two independent replicates. Numbers at the bottom of the plot indicate the number of samples in each group.\nextentry
				\subfig{C} Comparison of the enrichment of insulators and insulator fragments in insulator or silencer constructs. A linear regression line is shown as a solid line and its slope and goodness-of-fit ($R^2$) is indicated.\nextentry
				\subfig{D} Correlation between the slope of the regression lines from \parensubfig{C} and the strength of the corresponding enhancer from \parensubfig{B}. Pearson's $R^2$, Spearman's $\rho$, and number ($n$) of constructs are indicated. A linear regression line is shown as a dashed line.
			}%
			\label{fig:enhancers}%
		\end{fig}
	
	\fi
	%% Main figures end
	
	%%% Supplementary figures start
	\ifsupp
	
		\begin{sfig}
			\begin{tikzpicture}


	%%% scheme of the assay
	\coordinate (scheme) at (0, 0);
	
	% construct
	\coordinate[shift = {({.5 * (\fourcolumnwidth - 3.3cm)}, -1.25)}] (construct) at (scheme);
	
	\draw[line width = .2cm, -Triangle Cap, 35Senhancer]  ($(construct) + (.2, 0)$) -- ++(.6, 0) coordinate[pos = .5] (enh);
	
	\draw[Latex-] (enh) ++(0, -.15) -- ++(0, -.4) node[anchor = north, 35Senhancer, font = \bfseries] (lenh) {35S enhancer};

	\fill[insulator] (construct) ++(.9, -.2) -- ++(60:.5) coordinate (ins) -- ++(-60:.5) -- cycle;
	
	\draw[Latex-] (ins) ++(0, .05) -- ++(0, .4) node[anchor = south, insulator, font = \bfseries] {insulator};

	\draw[line width = .2cm, 35Spromoter] (construct) ++(1.5, 0) -- ++(.3, 0) coordinate[pos = .5] (pro);
	\draw[-{Stealth[round]}, semithick] (pro) ++(.15cm + .5\pgflinewidth, 0) |- ++(.35, .3);
	
	\draw[Latex-] (pro) ++(0, -.15) -- (pro |- lenh.south) node[anchor = north, 35Spromoter, font = \bfseries, align = center] (lpro) {35S minimal\\promoter};
	
	\node[draw = black, thin, anchor = west, minimum width = 1cm, outer xsep = 0pt] (GFP) at ($(pro) + (.45, 0)$) {GFP};
	
	\begin{pgfonlayer} {background}
		\fill[Orchid1] (GFP.north west) ++(.15, 0) rectangle ($(GFP.south west) + (.05, 0)$);
	\end{pgfonlayer}
	
	\coordinate[xshift = .2cm] (construct end) at (GFP.east);
	
	\begin{pgfonlayer} {background}
		\draw[thick, {Bar[width = .1cm]}-] (construct) -- (GFP.west);
		\draw[thick, -{Bar[width = .1cm]}] (GFP.east) -- (construct end);
	\end{pgfonlayer}
	
	% noEnh control
	\coordinate[shift = {(1.3cm, -1.1cm)}] (noEnh) at (construct |- lpro);

	\draw[line width = .2cm, 35Spromoter] (noEnh) ++(.2, 0) -- ++(.3, 0) coordinate[pos = .5] (pro);
	\draw[-{Stealth[round]}, semithick] (pro) ++(.15cm + .5\pgflinewidth, 0) |- ++(.35, .3);
	
	\node[draw = black, thin, anchor = west, minimum width = 1cm, outer xsep = 0pt] (GFP) at ($(pro) + (.45, 0)$) {GFP};
	
	\begin{pgfonlayer} {background}
		\fill[MediumPurple2] (GFP.north west) ++(.15, 0) rectangle ($(GFP.south west) + (.05, 0)$);
	\end{pgfonlayer}
	
	\coordinate[xshift = .2cm] (noEnh end) at (GFP.east);
	
	\begin{pgfonlayer} {background}
		\draw[thick, {Bar[width = .1cm]}-] (noEnh) -- (GFP.west);
		\draw[thick, -{Bar[width = .1cm]}] (GFP.east) -- (noEnh end);
	\end{pgfonlayer}
	
	\node[anchor = east, font = \bfseries] at (noEnh) {noEnh};
	
	% noIns control
	\coordinate[shift = {(.6cm, -.8cm)}] (noIns) at (construct |- noEnh);
	
	\draw[line width = .2cm, -Triangle Cap, 35Senhancer]  ($(noIns) + (.2, 0)$) -- ++(.6, 0) coordinate[pos = .5] (enh);

	\draw[line width = .2cm, 35Spromoter] (noIns) ++(.9, 0) -- ++(.3, 0) coordinate[pos = .5] (pro);
	\draw[-{Stealth[round]}, semithick] (pro) ++(.15cm + .5\pgflinewidth, 0) |- ++(.35, .3);
	
	\node[draw = black, thin, anchor = west, minimum width = 1cm, outer xsep = 0pt] (GFP) at ($(pro) + (.45, 0)$) {GFP};
	
	\begin{pgfonlayer} {background}
		\fill[HotPink1] (GFP.north west) ++(.15, 0) rectangle ($(GFP.south west) + (.05, 0)$);
	\end{pgfonlayer}
	
	\coordinate[xshift = .2cm] (noIns end) at (GFP.east);
	
	\begin{pgfonlayer} {background}
		\draw[thick, {Bar[width = .1cm]}-] (noIns) -- (GFP.west);
		\draw[thick, -{Bar[width = .1cm]}] (GFP.east) -- (noIns end);
	\end{pgfonlayer}
	
	\node[anchor = east, font = \bfseries] at (noIns) {noIns};
	
	
	%%% enrichment full-length insulators
	\coordinate (FLins) at (scheme -| \textwidth - \threequartercolumnwidth, 0);	
	
	\tikzset{
		rotate label/.code = {
			\ifnum3>#1\relax
				\pgfkeysalso{anchor = west, rotate = 90}
			\else
				\pgfkeysalso{anchor = south}
			\fi
		}
	}
	
	\begin{pggfplot}[%
		at = {(FLins)},
		total width = \threequartercolumnwidth,
		xlabel = insulator,
		ylabel = \enrichment,
		title is csname,
		title color from name,
		enlarge y limits = {lower, value = 0.15},
		xticklabels are csnames,
		xticklabel style = {font = \vphantom{\usename{lambda-EXOB}}}
	]{FLins_enrichment}
	
		\pggf_annotate(
			annotation = {
				\pgfplotsinvokeforeach{1, ..., 6}{
					\fill[gray, opacity = 0.1] (##1, \pggfymin) rectangle (##1 + 0.5, \pggfymax);
					\draw[gray, thin] (##1 + 0.5, \pggfymin) -- (##1 + 0.5, \pggfymax);
				}
				\fill[gray, opacity = 0.1] (7, \pggfymin) rectangle (\pggfxmax, \pggfymax);
			},
			zeroline
		)
		
		\pggf_hline()
	
		\pggf_boxplot(
			style from column = {fill = insulator},
			outlier options = {opacity = .5},
			outlier style from column = {insulator},
			label* = {
				node font = \figsmall\bfseries,
				yshift = .9\baselineskip,
				rotate label = #1,
			}{axis min}{sample},
		)
	
	\end{pggfplot}
	
	
	%%% subfigure labels
	\subfiglabel{scheme}
	\subfiglabel[anchor = south west]{scheme |- noEnh}
	\subfiglabel{FLins}

\end{tikzpicture}%
			\caption{%
				\captiontitle[\supports{\cref{fig:fragments}}]{Plant STARR-seq detects activity of enhancer-blocking insulators}%
				\subfig{A} Full-length insulators were cloned in the forward (fwd) or reverse (rev) orientation between a 35S enhancer and a 35S minimal promoter driving the expression of a barcoded GFP reporter gene.\nextentry
				\subfig{B} In all experiments, control constructs as in \parensubfig{A} but without an insulator (noIns) or without an insulator and without an enhancer (noEnh) were added to the library.\nextentry
				\subfig{C} All insulator constructs were pooled and subjected to Plant STARR-seq in \tobacco leaves (\tobacco) and \maize protoplasts (\maize). Reporter mRNA enrichment was normalized to a control construct without an enhancer or insulator (noEnh; log2 set to 0). Box plots represent the median (center line), upper and lower quartiles, and 1.5$\times$ interquartile range (whiskers) for all corresponding barcodes from two independent replicates. Numbers at the bottom of the plot indicate the number of samples in each group. The enrichment of a control construct without an insulator (noIns) is indicated as a dotted line.
			}%
			\label{sfig:FLins}
		\end{sfig}
		
		\begin{sfig}
			\begin{tikzpicture}
	
	%%% replicate correlation FLins library
	\coordinate (FLins) at (0, 0);
	
	\begin{pggfplot}[%
		at = {(FLins)},
		total width = \twocolumnwidth,
		xlabel = {\textbf{replicate 1:} \enrichment},
		ylabel = {\textbf{replicate 2:} \enrichment},
		col title is csname,
		col title color from name,
		row title = full-length insulator library,
		xytick = {-10, -8, ..., 10},
		legend columns = 2,
		legend style = {
			at = {(1, 0)},
			anchor = south east,
		},
		legend image post style = {opacity = 1},
		legend cell align = right,
		legend plot pos = right,
		scatter styles = {empty legend, noEnh, empty legend, noIns, lambda-EXOB, BEAD-1C, UASrpg, sIns1, sIns2, gypsy}
	]{FLins_cor_rep}
	
		\pggf_annotate(diagonal)
		
		\pggf_scatter(
			opacity = .5,
			style from column = insulator,
			noEnh/.style = {fill = noEnh, draw = noEnh, mark = diamond*},
			noIns/.style = {fill = noIns, draw = noIns, mark = diamond*}
		)
		
		\pggf_annotate(only facets = 2, scatter legend*)
			
		\pggf_stats(stats = {n,spearman,rsquare})
		
	\end{pggfplot}
	
	
	%%% replicate correlation iLib
	\coordinate (iLib) at (FLins -| \textwidth - \twocolumnwidth, 0);
	
	\begin{pggfplot}[%
		at = {(iLib)},
		total width = \twocolumnwidth,
		xlabel = {\textbf{replicate 1:} \enrichment},
		ylabel = {\textbf{replicate 2:} \enrichment},
		col title is csname,
		col title color from name,
		row title = insulator fragment library,
		xytick = {-10, -8, ..., 10},
		legend style = {
			at = {(1, 0)},
			anchor = south east,
		},
		legend image post style = {fill opacity = 1},
		legend cell align = right,
		legend plot pos = right,
		scatter styles = {none,35S,AB80,Cab-1}
	]{iLib_cor_rep}
	
		\pggf_annotate(diagonal)
	
		\pggf_scatter(
			fill opacity = .5,
			style from column = enhancer,
		)
	
		\pggf_annotate(only facets = 1, scatter legend*)
		\pggf_annotate(only facets = 2, legend = {\usename{none}, \usename{35S}})
	
		\pggf_stats(stats = {n,spearman,rsquare})
	
	\end{pggfplot}
	
	
	%%% replicate correlation iFC
	\coordinate[yshift = -\columnsep] (iFC) at (FLins |- xlabel.south);
	
	\begin{pggfplot}[%
		at = {(iFC)},
		total width = \twocolumnwidth,
		xlabel = {\textbf{replicate 1:} \enrichment},
		ylabel = {\textbf{replicate 2:} \enrichment},
		col title is csname,
		col title color from name,
		row title = {fragment combinations lib.},
		xytick = {-10, -8, ..., 10},
		facet 2/.append style = {pggf colorbar = {title = count, log2}}
	]{iFC_cor_rep}
	
		\pggf_annotate(diagonal)
	
		\pggf_hexbin(bins = 50)
	
		\pggf_stats(stats = {n,spearman,rsquare})
	
	\end{pggfplot}
	
	
	%%% replicate correlation iDE
	\coordinate (iDE) at (iFC -| \textwidth - \twocolumnwidth, 0);
	
	\distance{pggfplot c1r1.west}{pggfplot c1r1.east}
	
	\begin{pggfplot}[%
		at = {(iDE)},
		width = \xdistance,
		xlabel = {\textbf{replicate 1:} \enrichment},
		ylabel = {\textbf{replicate 2:} \enrichment},
		col title is csname,
		col title color from name,
		col title = tobacco,
		row title = downstream enhancer lib.,
		xytick = {-10, -8, ..., 10},
		legend style = {
			at = {(1, 0)},
			anchor = south east,
		},
		legend image post style = {fill opacity = 1},
		legend cell align = right,
		legend plot pos = right,
		scatter styles = {none,35S}
	]{iDE_cor_rep}
	
		\pggf_annotate(diagonal)
		
		\pggf_scatter(
			fill opacity = .5,
			style from column = enh_up
		)
	
		\pggf_annotate(scatter legend*)
	
		\pggf_stats(stats = {n,spearman,rsquare})
	
	\end{pggfplot}
	
	
	%%% replicate correlation i/sLib
	\coordinate[yshift = -\columnsep] (i/sLib) at (FLins |- xlabel.south);
	
	\begin{pggfplot}[%
		at = {(i/sLib)},
		total width = \twocolumnwidth,
		xlabel = {\textbf{replicate 1:} \enrichment},
		ylabel = {\textbf{replicate 2:} \enrichment},
		col title is csname,
		col title color from name,
		row title = insulator/silencer library,
		xytick = {-10, -8, ..., 10},
		legend style = {
			at = {(1, 0)},
			anchor = south east,
		},
		legend image post style = {fill opacity = 1},
		legend cell align = right,
		legend plot pos = right,
		scatter styles = {insulator,silencer}
	]{isLib_cor_rep}
	
		\pggf_annotate(diagonal)
		
		\pggf_scatter(
			fill opacity = .5,
			style from column = construct
		)
	
		\pggf_annotate(scatter legend*)
	
		\pggf_stats(stats = {n,spearman,rsquare})
	
	\end{pggfplot}
	
	
	%%% replicate correlation ExI
	\coordinate (ExI) at (i/sLib -| \textwidth - \twocolumnwidth, 0);
	
	\begin{pggfplot}[%
		at = {(ExI)},
		total width = \twocolumnwidth,
		xlabel = {\textbf{replicate 1:} \enrichment},
		ylabel = {\textbf{replicate 2:} \enrichment},
		col title is csname,
		col title color from name,
		row title = enhancer-insulator combi,
		xytick = {-10, -8, ..., 10},
		legend style = {
			at = {(1, 0)},
			anchor = south east,
		},
		legend image post style = {fill opacity = 1},
		legend cell align = right,
		legend plot pos = right,
		scatter styles = {insulator,silencer}
	]{ExI_cor_rep}
	
		\pggf_annotate(diagonal)
		
		\pggf_scatter(
			fill opacity = .5,
			style from column = construct
		)
	
		\pggf_annotate(scatter legend*)
	
		\pggf_stats(stats = {n,spearman,rsquare})
	
	\end{pggfplot}
	
	
	%%% subfigure labels
	\subfiglabel{FLins}
	\subfiglabel{iLib}
	\subfiglabel{iFC}
	\subfiglabel{iDE}
	\subfiglabel{i/sLib}
	\subfiglabel{ExI}
	
\end{tikzpicture}
%
			\caption{%
				\captiontitle[\supports{all figures}]{Plant STARR-seq yields highly reproducible results}%
				\subfigrange{A}{F} Correlation between biological replicates of Plant STARR-seq for the full-length insulator library used in \cref{sfig:FLins} \parensubfig{A}, the insulator fragment library used in \cref{fig:fragments} \parensubfig{B}, the insulator fragment combination library used in \cref{fig:stacking} \parensubfig{C}, the downstream enhancer library \parensubfig{D} and the insulator/silencer library \parensubfig{E} used in \cref{fig:silencer}, and the enhancer-insulator combination library used in \cref{fig:enhancers} \parensubfig{F}. Experiments were performed in \tobacco leaves (\tobacco) or \maize protoplasts (\maize) as indicated. Pearson's $R^2$, Spearman's $\rho$, and number ($n$) of constructs are indicated. The color in the hexbin plots in \parensubfig{C} represents the count of points in each hexagon.
			}%
			\label{sfig:PEfl_ori}
		\end{sfig}
		
		\begin{sfig}
			\begin{tikzpicture}

	%%% maize ELISA frags: boxplots
	\coordinate (ELISA) at (0, 0);
	
	\begin{pggfplot}[%
		at = {(ELISA)},
		total width = \twocolumnwidth,
		ylabel = ELISA (ppm/TP),
		ylabel style = {yshift = .5\baselineskip},
		ylabel add to width = \baselineskip,
		xticklabels are csnames,
		xticklabel style = {font = \vphantom{\usename{lambda-EXOB}}, rotate = 45, anchor = north east, align = right},
		title is csname,
		title color from name,
		enlarge y limits = {lower, value = 0.1},
		tight y,
	]{ELISA_samples}
	
		\pggf_annotate(zeroline)
		
		\pggf_hline(
			noEnh/.style = dashed,
			noIns/.style = {draw = black},
			style from column = intercept_id
		)
	
		\pggf_boxplot(
			style from column = {fill = xticklabels},
		)
	
	\end{pggfplot}
	
	
	%%% maize ELISA frags: correlation
	\coordinate (cor) at (ELISA -| \textwidth - \twocolumnwidth, 0);
	
	\begin{pggfplot}[%
		at = {(cor)},
		total width = \threecolumnwidth,
		xlabel = \textbf{Plant STARR-seq:} \enrichment,
		ylabel = \textbf{stable lines:} ELISA (ppm/TP),
		ylabel style = {yshift = .5\baselineskip},
		ylabel add to width = \baselineskip,
		title is csname,
		title color from name,
		legend columns = 1,
		legend style = {
			at = {(1, 1)},
			anchor = west,
			cells = {align = right},
			shift = {(2\baselineskip, .5\groupplotsep)}
		},
		legend cell align = left,
		legend plot pos = left,
		scatter styles = {noEnh,noIns,beta-phaseolin_781_612,beta-phaseolin_1395_1564,beta-phaseolin_2356_2187,TBS_588_757,lambda-EXOB_666_497,sIns2_335_504}
	]{ELISA_cor_samples}
	
		\pggf_trendline()
	
		\pggf_scatter(
			mark size = 2,
			style from column = id,
			y error = CI
		)
		
		\pggf_annotate(
			scatter legend*,
			only facets = 3
		)
	
		\pggf_stats(stats = {n,spearman,rsquare})
	
	\end{pggfplot}
	
	
	%%% subfigure labels
	\subfiglabel{ELISA}
	\subfiglabel{cor}

\end{tikzpicture}%
			\caption{%
				\captiontitle[\supports{\cref{fig:insDL}}]{Activity of insulator fragments in different maize tissues}%
				\subfigtwo{A}{B} Transgenic \maize lines were created using constructs as in \cref{fig:insDL}\subfigunformatted{h}. The activity of insulator fragments was measured in the indicated tissues \parensubfig{A} and compared to the corresponding results from Plant STARR-seq in \maize protoplasts \parensubfig{B}.
				Box plots in \parensubfig{A} represent the median (center line), upper and lower quartiles (box limits), 1.5$\times$ interquartile range (whiskers), and outliers (points) for all corresponding samples from two to three independent replicates. Numbers at the bottom of each box plot indicate the number of samples in each group. For groups with less than 10 samples, individual data points are shown as black dots. The dotted and dashed lines in \parensubfig{A} represent the median enrichment of control constructs without an insulator or without an enhancer, respectively. In \parensubfig{B}, the dashed line represents a linear regression line and error bars represent the 95\% confidence interval. Pearson's $R^2$, Spearman's $\rho$, and number ($n$) of constructs are indicated.
			}%
			\label{sfig:ELISA_frags}%
		\end{sfig}
		
		\begin{sfig}
			\begin{tikzpicture}

	%%% maize ELISA frags: boxplots
	\coordinate (ELISA) at (0, 0);
	
	\begin{pggfplot}[%
		at = {(ELISA)},
		total width = \twocolumnwidth,
		ylabel = ELISA (ppm/TP),
		ylabel style = {yshift = .5\baselineskip},
		ylabel add to width = \baselineskip,
		xticklabels are csnames,
		xticklabel style = {font = \vphantom{\usename{lambda-EXOB}}, rotate = 45, anchor = north east, align = right},
		title is csname,
		title color from name,
		enlarge y limits = {lower, value = 0.1},
		tight y,
	]{ELISA_stacked_samples}
	
		\pggf_annotate(zeroline)
		
		\pggf_hline(
			noEnh/.style = dashed,
			noIns/.style = {draw = black},
			style from column = intercept_id
		)
	
		\pggf_boxplot(
			style from column = {fill = xticklabels},
		)
	
	\end{pggfplot}
	
	
	%%% maize ELISA frags: correlation
	\coordinate (cor) at (ELISA -| \textwidth - \twocolumnwidth, 0);
	
	\begin{pggfplot}[%
		at = {(cor)},
		total width = \threecolumnwidth,
		xlabel = \textbf{Plant STARR-seq:} \enrichment,
		ylabel = \textbf{stable lines:} ELISA (ppm/TP),
		ylabel style = {yshift = .5\baselineskip},
		ylabel add to width = \baselineskip,
		title is csname,
		title color from name,
		legend columns = 1,
		legend style = {
			at = {(1, 1)},
			anchor = west,
			cells = {align = right},
			shift = {(2\baselineskip, .5\groupplotsep)}
		},
		legend cell align = left,
		legend plot pos = left,
		scatter styles = {noEnh,noIns,D2,T30,T21,T27,T32,T24,T25,T19,T9}
	]{ELISA_stacked_cor_samples}
	
		\pggf_trendline()
	
		\pggf_scatter(
			mark size = 2,
			style from column = short_id,
			y error = CI
		)
		
		\pggf_annotate(
			scatter legend*,
			only facets = 3
		)
	
		\pggf_stats(
			stats = {n,spearman,rsquare},
			position = south east
		)
	
	\end{pggfplot}
	
	
	%%% subfigure labels
	\subfiglabel{ELISA}
	\subfiglabel{cor}

\end{tikzpicture}%
			\caption{%
				\captiontitle[\supports{\cref{fig:stacking}}]{Activity of insulator fragment combinations in different maize tissues}%
				\subfigtwo{A}{B} Transgenic \maize lines were created using insulator fragment combinations in constructs as in \cref{fig:insDL}\subfigunformatted{h}. The activity of insulator fragments was measured in the indicated tissues \parensubfig{A} and compared to the corresponding results from Plant STARR-seq in \maize protoplasts \parensubfig{B}. In \parensubfig{A}, box plots are as defined in \cref{sfig:ELISA_frags}, and the dotted and dashed lines represent the median enrichment of control constructs without an insulator or without an enhancer, respectively. In \parensubfig{C}, the dashed line represents a linear regression line and error bars represent the 95\% confidence interval. Pearson's $R^2$, Spearman's $\rho$, and number ($n$) of constructs are indicated.
			}%
			\label{sfig:ELISA_iFC}%
		\end{sfig}
		
		\begin{sfig}
			\begin{tikzpicture}

	%%% correlation iDE and iLib libraries
	\coordinate (correlation) at (0, 0);
	
	\begin{pggfplot}[%
		at = {(correlation)},
		total width = \twocolumnwidth,
		xlabel = {\textbf{with downstream enhancer:} \enrichment},
		ylabel = {\textbf{without downstream enhancer:}\\\enrichment},
		title is csname,
		title color from name,
		ylabel add to width = \baselineskip,
		xytick = {-10, ..., 10}
	]{iDE_cor_iLib}
	
		\pggf_scatter(
			data = iDE_cor_iLib_fragments,
			color from col name,
			opacity = .5
		)
		
		\pggf_scatter(
			data = iDE_cor_iLib_control,
			mark size = 2.5
		)
		
		\pggf_stats(stats = {n,spearman,rsquare})
	
		\pggf_abline(gray, dashed)
	
	\end{pggfplot}

\end{tikzpicture}%
			\caption{%
				\captiontitle[\supports{\cref{fig:silencer}}]{Enhancers downstream of insulator fragments slightly reduce their activity}%
				Correlation between the activity of insulator fragments cloned between a 35S enhancer and a 35S minimal promoter with or without an additional \usename{AB80} or \usename{Cab-1} enhancer inserted between the insulator fragment and 35S minimal promoter. The dashed line represents a y = x line fitted through the point corresponding to a control construct without an insulator (black dot). Pearson's $R^2$, Spearman's $\rho$, and number ($n$) of constructs are indicated.
			}%
			\label{sfig:iDE_vs_iLib}%
		\end{sfig}
		
		\begin{sfig}
			\begin{tikzpicture}

	%%% dual-luciferase construct
	\coordinate (DL construct) at (0, 0);
	
	\coordinate[shift = {(.2, -1.25)}] (construct) at (DL construct);
	
	\node[draw = black, thin, anchor = west, text depth = 0pt] (Luc) at ($(construct) + (.2, 0)$) {Luc};
	
	\draw[line width = .2cm, DarkOliveGreen3] (Luc.east) ++(.3, 0) coordinate (c1) -- ++(.5, 0) coordinate[pos = .5] (pro1) coordinate (c2);
	\draw[-{Stealth[round]}, thick] (c1) ++(-.5\pgflinewidth, 0) |- ++(-.35, .3);
	
	\node[draw = black, thin, anchor = west, text depth = 0pt] (BlpR) at ($(c2) + (.2, 0)$) {B/H\textsuperscript{R}};
	
	\draw[line width = .2cm, PaleGreen3] (BlpR.east) ++(.35, 0) coordinate (c3) -- ++(.5, 0) coordinate (c4);
	\draw[-{Stealth[round]}, thick] (c3) ++(-.5\pgflinewidth, 0) |- ++(-.35, .3);
	
	\fill[insulator] (c4) ++(.4, -.2) -- ++(60:.5) coordinate (ins) -- ++(-60:.5) -- cycle;
	
	\draw[line width = .2cm, -Triangle Cap, 35Senhancer] (c4) ++(1, 0) -- ++(.6, 0) coordinate[pos = .5] (enh) coordinate (c5);
	
	\draw[line width = .2cm, 35Spromoter] (c5) ++(.1, 0) -- ++(.3, 0) coordinate[pos = .5] (pro2) coordinate (c6);
	\draw[-{Stealth[round]}, thick] (c6) ++(.5\pgflinewidth, 0) |- ++(.35, .3);
	
	\node[draw = black, thin, anchor = west, text depth = 0pt] (NLuc) at ($(c6) + (.3, 0)$) {NanoLuc};
	
	\coordinate[xshift = .2cm] (construct end) at (NLuc.east);
	
	\begin{pgfonlayer} {background}
		\draw[thick, {Bar[width = .1cm]}-] (construct) -- (Luc.west);
		\draw[thick] (Luc.east) -- (BlpR.west) (BlpR.east) -- (NLuc.west);
		\draw[thick, -{Bar[width = .1cm]}] (NLuc.east) -- (construct end);
	\end{pgfonlayer}
	
	\draw[Latex-] (ins) ++(0, .05) -- ++(0, .4) node[anchor = south, font = \bfseries] {insulator};
	\draw[Latex-] (pro1) ++(0, -.15) -- ++(0, -.4) node[anchor = north] {\textit{AtUBQ10} promoter};
	\draw[Latex-] (enh) ++(0, -.15) -- ++(-.3, -.4) coordinate (lenh);
	\node[anchor = north, align = center, xshift = -.25cm] at (lenh) {enhancer:\\\textcolor{35S}{\bfseries\usename{35S}} or \textcolor{AB80}{\bfseries\usename{AB80}}};
	\draw[Latex-] (pro2) ++(0, -.15) -- ++(.3, -.4) coordinate (lpro);
	\node[anchor = north, align = center, xshift = .25cm] (l35Spr) at (lpro) {35S minimal\\promoter};

	\draw[decorate, decoration = {brace, raise = -.15cm, amplitude = .4cm, aspect = .825}] (l35Spr.south -| construct end) -- (l35Spr.south -| construct) coordinate[pos = .825, yshift = -.25cm] (curly);
	
	\draw[ultra thick, -Latex] (curly) ++(0, .75\pgflinewidth) -- ++(0, -.3) arc[start angle = 180, end angle = 270, radius = .8cm] -- ++(.3, 0) coordinate (plants);
	\planticon[scale = .66]{arabidopsis} at ($(plants) + (.7, 0)$);
	\node [anchor = north east, align = center, inner xsep = 0pt] at (curly) {integrate\\into\\\Arabidopsis\\genome};
	
	\draw[ultra thick, -Latex] (plants) ++(1.4, 0) -- ++(.8, 0) node[anchor = west, align = center] {measure\\luciferase (Luc)\\and nanoluciferase\\(NanoLuc) activity};


	%%% DL silencer constructs: boxplots
	\coordinate[xshift = \columnsep] (boxplots) at (DL construct -| construct end);
	
	\distance{boxplots}{\textwidth, 0}
	
	\begin{pggfplot}[%
		at = {(boxplots)},
		total width = \xdistance,
		ylabel = \dlratio,
		xticklabels are csnames,
		xticklabel style = {font = \vphantom{\usename{lambda-EXOB}}, rotate = 45, anchor = north east},
		row title = Arabidopsis,
		title is csname,
		title color from name,
		enlarge y limits = {lower, value = 0.1},
		tight y,
		ylabel add to width = .5\baselineskip,
	]{DL_sil}
		
		\pggf_annotate(zeroline)
		
		\pggf_hline()
	
		\pggf_boxplot(
			style from column = {fill = xticklabels},
			outlier options = {opacity = .5},
			outlier style from column = {xticklabels},
		)
	
	\end{pggfplot}
	
	
	%%% DL silencer constructs: correlation
	\coordinate[yshift = -.5\columnsep] (correlation) at (DL construct |- pggfplot c1r1.below south);
	
	\begin{pggfplot}[%
		at = {(correlation)},
		total width = \twocolumnwidth,
		xlabel = \textbf{Plant STARR-seq:} \enrichment,
		ylabel = \textbf{stable lines:} \dlratio,
		row title = Arabidopsis/tobacco,
		title is csname,
		col title color from name,
		row title style from name,
		row title style = {shading angle = 0},
		legend columns = 2,
		legend style = {
			at = {(1, 0)},
			anchor = south east,
		},
		legend cell align = right,
		legend plot pos = right,
		scatter styles = {empty legend, noEnh, empty legend, noIns, empty legend, lambda-EXOB, empty legend, BEAD-1C, UASrpg, sIns1, sIns2, gypsy}
	]{DL_cor_sil}
	
		\pggf_trendline()
	
		\pggf_scatter(
			mark size = 2,
			style from column = insulator,
			y error = CI,
			noEnh/.style = {fill = noEnh, draw = noEnh, mark = diamond*},
			noIns/.style = {fill = noIns, draw = noIns, mark = diamond*}
		)
	
		\pggf_annotate(
			scatter legend*,
			only facets = 2
		)
	
		\pggf_stats(stats = {n,spearman,rsquare})
	
	\end{pggfplot}
	
	
	%%% DL insulator vs. silencer
	\coordinate[xshift = \columnsep] (IvS) at (pggfplot c2r1.outer north east);
	
	\begin{pggfplot}[%
		at = {(IvS)},
		total width = \twocolumnwidth,
		xlabel = {\textbf{\usename{insulator}:} \enrichment},
		ylabel = {\textbf{\usename{silencer}:} \enrichment},
		row title = Arabidopsis,
		title is csname,
		title color from name,
		xytick = {-20, -18, ..., 20}
	]{DL_IvS}
	
		\pggf_annotate(diagonal)
	
		\pggf_trendline(
			thick,
			solid
		)
	
		\pggf_stats(
			stats = {slope,goodness of fit},
			position = south east
		)
	
		\pggf_scatter(
			mark size = 1.5,
			color from col name
		)
	
	\end{pggfplot}


	%%% subfigure labels
	\subfiglabel{DL construct}
	\subfiglabel{boxplots}
	\subfiglabel{correlation}
	\subfiglabel{IvS}

\end{tikzpicture}%
			\caption{%
				\captiontitle[\supports{\cref{fig:enhancers}}]{Enhancer-dependent silencer activity in stable transgenic plants}%
				\subfig{A} Transgenic \Arabidopsis lines were generated with T-DNAs harboring a constitutively expressed luciferase (Luc) gene and a nanoluciferase (NanoLuc) gene under control of a 35S minimal promoter coupled to the \usename{35S} or \usename{AB80} enhancer (as indicated above the plots) with insulator candidates inserted upstream of the enhancer. Nanoluciferase activity was measured in at least 4 plants from these lines and normalized to the activity of luciferase. The NanoLuc/Luc ratio was normalized to a control construct without an enhancer or insulator (noEnh; log2 set to 0).\nextentry
				\subfigtwo{B}{C} The activity of full-length insulators was measured in \Arabidopsis lines \parensubfig{B} and compared to the corresponding results from Plant STARR-seq in \tobacco leaves \parensubfig{C}. Box plots in \parensubfig{B} are as defined in \cref{sfig:ELISA_frags} and the dotted line indicates the median activity of a control construct without an insulator. In \parensubfig{C}, the dashed line represents a linear regression line and error bars represent the 95\% confidence interval. Pearson's $R^2$, Spearman's $\rho$, and number ($n$) of constructs are indicated.\nextentry
				\subfig{D} Comparison of the mean NanoLuc/Luc ratio of full-length insulators in insulator (\cref{fig:insDL}\subfigunformatted{B}) or silencer constructs \parensubfig{B}. A linear regression line is shown as a solid line and its slope and goodness-of-fit ($R^2$) is indicated.
			}%
			\label{sfig:silDL}
		\end{sfig}

	\fi
	%%% Supplementary figures end
	
	%%% Supplemenary tables start
	\ifstab
	
		\begin{stab}
			\caption{%
				\captiontitle{Insulator fragments used in fragment combination library}%
				Positions are numbered by increasing distance from the minimal promoter (position 1 is the fragment closest to the promoter, position 3 the most distal one).
			}%
			\label{stab:iFC_frags}%
			\rowcolors{2}{white}{gray!10}

\begin{tabularx}{\textwidth}{XXXXl}
	\toprule
	\textbf{insulator} & \textbf{start} & \textbf{stop} & \textbf{orientation} & \textbf{insulator activity in \textit{N. benthamiana}} \\
	\midrule
	\multicolumn{5}{c}{\textbf{fragments for position 1 and 2}} \\
	\midrule
	\usename{beta-phaseolin} & 230 & 399 & fwd and rev & bottom 25\% (both orientations) \\
	\usename{beta-phaseolin} & 383 & 552 & fwd and rev & bottom 25\% (both orientations) \\
	\usename{beta-phaseolin} & 1148 & 1317 & fwd and rev & top 25\% (rev orientation) \\
	\usename{beta-phaseolin} & 1317 & 1486 & fwd and rev & top 25\% (rev orientation) \\
	\usename{beta-phaseolin} & 1395 & 1564 & fwd and rev & top 25\% (both orienations) \\
	\usename{beta-phaseolin} & 1633 & 1802 & fwd and rev & top 25\% (fwd orientation) \\
	\usename{beta-phaseolin} & 1712 & 1881 & fwd and rev & top 25\% (both orienations) \\
	\usename{beta-phaseolin} & 1791 & 1960 & fwd and rev & top 25\% (both orienations) \\
	\usename{beta-phaseolin} & 2266 & 2435 & fwd and rev & top 25\% (both orienations) \\
	\usename{beta-phaseolin} & 2345 & 2514 & fwd and rev & top 25\% (fwd orientation) \\
	\usename{beta-phaseolin} & 2741 & 2910 & fwd and rev & top 25\% (fwd orientation) \\
	\usename{beta-phaseolin} & 3058 & 3227 & fwd and rev & top 25\% (both orienations) \\
	\usename{beta-phaseolin} & 3454 & 3623 & fwd and rev & bottom 25\% (both orientations) \\
	\usename{TBS} & 252 & 421 & fwd and rev & top 25\% (both orienations) \\
	\usename{TBS} & 588 & 757 & fwd and rev & top 25\% (both orienations) \\
	\usename{TBS} & 756 & 925 & fwd and rev & bottom 25\% (both orientations) \\
	\usename{TBS} & 1681 & 1850 & fwd and rev & top 25\% (both orienations) \\
	\usename{TBS} & 1765 & 1934 & fwd and rev & top 25\% (rev orientation) \\
	\usename{lambda-EXOB} & 1 & 170 & fwd and rev & top 25\% (both orienations) \\
	\usename{lambda-EXOB} & 83 & 252 & fwd and rev & top 25\% (both orienations) \\
	\usename{lambda-EXOB} & 166 & 335 & fwd and rev & top 25\% (both orienations) \\
	\usename{lambda-EXOB} & 249 & 418 & fwd and rev & top 25\% (both orienations) \\
	\usename{lambda-EXOB} & 332 & 501 & fwd and rev & top 25\% (both orienations) \\
	\usename{lambda-EXOB} & 415 & 584 & fwd and rev & top 25\% (both orienations) \\
	\usename{lambda-EXOB} & 663 & 832 & fwd and rev & top 25\% (rev orientation) \\
	\usename{lambda-EXOB} & 829 & 998 & fwd and rev & top 25\% (fwd orientation) \\
	\usename{BEAD-1C} & 246 & 415 & fwd and rev & top 25\% (both orienations) \\
	\usename{UASrpg} & 157 & 326 & fwd and rev & top 25\% (fwd orientation) \\
	\usename{sIns1} & 54 & 223 & fwd and rev & top 25\% (rev orientation) \\
	\usename{sIns2} & 335 & 504 & fwd and rev & top 25\% (both orienations) \\
	\usename{gypsy} & 1 & 170 & fwd and rev & bottom 25\% (both orientations) \\
	\usename{gypsy} & 54 & 223 & fwd and rev & bottom 25\% (both orientations) \\
	\midrule
	\multicolumn{5}{c}{\textbf{fragments for position 3}} \\
	\midrule
	\usename{beta-phaseolin} & 1633 & 1802 & fwd & top 5\% \\
	\usename{beta-phaseolin} & 1712 & 1881 & rev & top 5\% \\
	\usename{lambda-EXOB} & 1 & 170 & fwd & top 5\% \\
	\usename{lambda-EXOB} & 332 & 501 & fwd & top 5\% \\
	\usename{lambda-EXOB} & 415 & 584 & rev & top 5\% \\
	\bottomrule
\end{tabularx}%
		\end{stab}
		
		\begin{stab}
			\caption{%
				\captiontitle{Insulator fragment combinations tested in stable transgenic maize lines}%
				Positions are numbered by increasing distance from the minimal promoter (position 1 is the fragment closest to the promoter, position 3 the most distal one).
			}%
			\label{stab:iFC_maize}%
			\rowcolors{2}{white}{gray!10}

\begin{tabularx}{\textwidth}{llXXXl}
	\toprule
	\textbf{name} & \textbf{fragments} & \textbf{position 3} & \textbf{position 2} & \textbf{position 1} & \textbf{insulator activity in maize} \\
	\midrule
	D2 & 2 &  & \usename{beta-phaseolin}\newline 1148-1317, fwd & \usename{sIns2}\newline 335-504, fwd & strong \\
	T30 & 3 & \usename{beta-phaseolin}\newline 1633-1802, fwd & \usename{lambda-EXOB}\newline 663-832, fwd & \usename{beta-phaseolin}\newline 1564-1395, rev & intermediate \\
	T21 & 3 & \usename{beta-phaseolin}\newline 1881-1712, rev & \usename{lambda-EXOB}\newline 663-832, fwd & \usename{lambda-EXOB}\newline 170-1, rev & weak \\
	T27 & 3 & \usename{lambda-EXOB}\newline 1-170, fwd & \usename{lambda-EXOB}\newline 584-415, rev & \usename{TBS}\newline 925-756, rev & strong \\
	T32 & 3 & \usename{lambda-EXOB}\newline 584-415, rev & \usename{lambda-EXOB}\newline 832-663, rev & \usename{beta-phaseolin}\newline 1960-1791, rev & strong \\
	T24 & 3 & \usename{lambda-EXOB}\newline 584-415, rev & \usename{beta-phaseolin}\newline 3227-3058, rev & \usename{lambda-EXOB}\newline 170-1, rev & strong \\
	T25 & 3 & \usename{lambda-EXOB}\newline 1-170, fwd & \usename{beta-phaseolin}\newline 1802-1633, rev & \usename{beta-phaseolin}\newline 1317-1148, rev & strong \\
	T19 & 3 & \usename{lambda-EXOB}\newline 1-170, fwd & \usename{beta-phaseolin}\newline 1148-1317, fwd & \usename{sIns2}\newline 335-504, fwd & strong \\
	T9 & 3 & \usename{lambda-EXOB}\newline 584-415, rev & \usename{TBS}\newline 1765-1934, fwd & \usename{sIns2}\newline 335-504, fwd & strong \\
	\bottomrule
\end{tabularx}
%
		\end{stab}
	
		\begin{stab}
			\caption{%
				\captiontitle{Insulators and insulator fragments used in the enhancer-insulator combination library}%
			}%
			\label{stab:ExI}%
			\rowcolors{2}{white}{gray!10}

\begin{tabularx}{\textwidth}{XXXXX}
	\toprule
	\textbf{type} & \textbf{insulator} & \textbf{start} & \textbf{stop} & \textbf{orientation} \\
	\midrule
	full-length insulator & \usename{lambda-EXOB} & 1 & 998 & fwd \\
	full-length insulator & \usename{BEAD-1C} & 1 & 538 & fwd \\
	full-length insulator & \usename{UASrpg} & 1 & 378 & fwd \\
	full-length insulator & \usename{sIns1} & 1 & 386 & fwd \\
	full-length insulator & \usename{sIns2} & 1 & 504 & fwd \\
	full-length insulator & \usename{gypsy} & 1 & 386 & fwd \\
	insulator fragment & \usename{beta-phaseolin} & 1395 & 1564 & fwd \\
	insulator fragment & \usename{TBS} & 756 & 925 & fwd \\
	insulator fragment & \usename{lambda-EXOB} & 1 & 170 & fwd \\
	insulator fragment & \usename{BEAD-1C} & 246 & 415 & fwd \\
	insulator fragment & \usename{UASrpg} & 1 & 170 & fwd \\
	insulator fragment & \usename{gypsy} & 54 & 223 & fwd \\
	\bottomrule
\end{tabularx}%
		\end{stab}
		
		\begin{longstab}
			\begin{xltabular}{\textwidth}{Xl}
	\caption{
		\captiontitle{Full-length insulator sequences}
	}\label{stab:FLins}\\
	\toprule
	\textbf{insulator} & \textbf{sequence} \\
	\midrule
	\endfirsthead
	\toprule
	\textbf{insulator} & \textbf{sequence} \\
	\midrule
	\endhead
	\midrule
	\endfoot
	\bottomrule
	\endlastfoot
	\rowcolor{gray!10}\usename{beta-phaseolin} & 
		\begin{minipage}{.85\textwidth}
			\vspace*{4pt}
			\DNA!'{gray!10}
				CATAAGAAATATGAAAATCGTTATGAACTTTATTATTTGTTAAACGTTTTCATAACCGCATAAAA
				TTTTATAAAGTCCCGTCTATCTTTAATATGTAGTCTAACATTTTCATATTGAAATATATAATTTA
				CTTAATTTTAGTGTTGGTAGAAAGCATAATGATTTATTCTTGTTCATATAAATGTTTAATATAAA
				CAAACTCTTTACCTTAAGAAGGATTTCCCATTTTATATTTTAAAAATATATTTATCAAATATTTT
				TCAACCACGTAAATCTCATAATAATAAGTTGTTTCAAAAGTAATAAAATTTAACTCCATAATTTT
				TTTATTTGACTGATCTTAAAGCAACACCCAGTGACACAACTAGTAATTTTTTTCTTTGAATAAAA
				AAATCCAATTATCATTGTATTTTTTTTATACAATGAAAATTTCACCAAACAATGATTTGTGGTAT
				TTCTGAAGCAAGTCATGTTAATGCAAAATTCTATAATTCACATTTGACACTACGGAAGTGACTGA
				AAATCTGTTTTTACATGCGAGACACATCAATTTTTAATTCTAAAGTAATTTTAATAATAGTTACT
				ATATTCAAGATTTGATATATCAAATACTCAATATTACTTCTAAAAAATTAATTAGATATAATTAA
				AAAATTACTTTTTTAATTTTAAGTTTAATTGTTGGATTTGTGACTATTGATTTATTATTCTACTA
				TGTTTAAACTGTTTTATAGATAGTTTAAAGTAAATATAAGTATTGTAGAGTGTTACCGTAAACTA
				TAAGATTTATGTTGGACTAATTTTATGTTCTTCATTTGCAATATTTTAATATATTTGTTGTTGGT
				TTACCTTTCTTGGTATGTAAGTCCGTAACCAGAATTACTGTGGGTTGCCATGGCACTCTGTAGTC
				TTTTGGTTCGTGCATGGATGCTTGCGCAAGAAAAAGACATAGAACAAAGAAAAAAGACAAAACAG
				AGAGAGAAAACGCAATCACACAACCAACTCAAATTAGTCACTGGCTGATCAAGATCGCCGCGTCC
				ATGTATGTCTAAATGCCATGCAAAGCAACACGTGCTTAACATGCACTTTAAATGGCTCACCCATC
				TCAACCCACACACAAACACATTGTCTTTTTCTTCATCATCACCACAACCACCTGTATATATTCAT
				TCTCTTCCGCCACCTCAATTTCTTCACTTCAACACACGTCAACCTGCATATGCGTGTCATCCCAT
				GCCCAAATCTCCATGCATGTTCCAACCACCTTCTCTCTTATATAATACCTATAAATACCCCTAAT
				ATCACTCACTTCTTTCATCATCCATCCATCCAGAGTACTACTACTCTACTACTATAATACCCCAA
				CCCAACTCATATTCAATACTACTCTACTATGATGAGAGCAAGGGTTCCACTCCTGTTGCTGGGAA
				TTCTTTTCCTGGCATCACTTTCTGCCTCATTTGCCACTTCACTCCGGGAGGAGGAAGAGAGCCAA
				GATAACCCCTTCTACTTCAACTCTGACAACTCCTGGAACACTCTATTCAAAAACCAATATGGTCA
				CATTCGTGTCCTCCAGAGGTTCGACCAACAATCCAAACGACTTCAGAATCTTGAAGACTACCGTC
				TTGTGGAGTTCAGGTCCAAACCCGAAACCCTCCTTCTTCCTCAGCAGGCTGATGCTGAGTTACTC
				CTAGTTGTCCGTAGTGGTAAGTAATTGCTACTGGTATCACTTGTTTCTTCTTGCAGAAATAATGG
				TAATGAGTTTTTTTATAATTTCAGGGAGCGCCATACTCGTCTTGGTGAAACCTGATGATCGCAGA
				GAGTACTTCTTCCTTACGCAAGGCGATAACCCGATATTCTCTGATAACCAGAAAATCCCTGCAGG
				AACCATTTTCTATTTGGTTAACCCTGACCCCAAAGAGGATCTCAGAATAATCCAACTCGCCATGC
				CCGTTAACAACCCTCAGATTCATGTACTGCCTTTTGTAATACCAAACTAATTTTTTTGTTATTTT
				AACTTGCAATTTCTCTCCAAATGTGATGATAAATGTTTGTCCTGTAGGAATTTTTCCTATCTAGC
				ACAGAAGCCCAACAATCCTACTTGCAAGAGTTCAGCAAGCATATTCTAGAGGCCTCCTTCAATGT
				AAGAAAGAAAACAGCATCTAACTACATATTTGCGTCATCTAACTACATATTTTCGTTGCCATTTA
				GCTAGTACTTTGTCTAAATGTCACACTTGTTGAATTTGTTGAATGATATCATTATATATGTTTGC
				ATGATTTTTATAGAGCAAATTCGAGGAGATCAACAGGGTTCTGTTTGAAGAGGAGGGACAGCAAG
				AGGGAGTGATTGTGAACATTGATTCTGAACAGATTGAGGAACTGAGCAAACATGCAAAATCTAGT
				TCAAGGAAATCCCATTCCAAACAAGATAACACAATTGGAAACGAATTTGGAAACCTGACTGAGAG
				GACCGATAACTCCTTGAATGTGTTAATCAGTTCTATAGAGATGAAAGAGGTAAATACAAAGAAAA
				AACATATAGACAAACTTAGCAATTGAGTTCTATTATTCACTGTCGTCTTGGTTAGAAAATCTTAG
				TATTGAGAATATAATTAAATAATGGTTTTTTTTGTTAACAAATTTAGGGAGCTCTTTTTGTGCCA
				CACTACTATTCTAAGGCCATTGTTATACTAGTGGTTAATGAAGGAGAAGCACATGTTGAACTTGT
				TGGCCCAAAAGGAAATAAGGAAACCTTGGAATTTGAGAGCTACAGAGCTGAGCTTTCTAAAGACG
				ATGTATTTGTAATCCCAGCAGCATATCCAGTTGCCATCAAGGCTACCTCCAACGTGAATTTCACT
				GGTTTCGGTATCAATGCTAATAACAACAATAGGAACCTCCTTGCAGGTATATATATTTATTATAT
				ATGACCATGAATTTGAATATAGGGTTGTTGATGGGATTTTTTATTTATAATTGGTAATGCGTGAT
				TGTGATTGAAAATATGAAGGTAAGACGGACAATGTCATAAGCAGCATCGGTAGAGCTCTGGACGG
				TAAAGACGTGTTGGGGCTTACGTTCTCTGGGTCTGGTGAAGAAGTTATGAAGCTGATCAACAAGC
				AGAGTGGATCGTACTTTGTGGATGGACACCATCACCAACAGGAACAGCAAAAGGGAAGTCACCAA
				CAGGAACAGCAAAAGGGAAGAAAGGGTGCATTTGTGTACTGAATAAGTATGAACTAAAATGCATG
				TATGGTGTAAGAGCTCATGGAGAGCATGGAAATATGTATCAGACCATGTAACACTATAATAACTG
				AGCTCCATCTCACTTCTTCTATGAATAAACAAAGGATGTTATGATATATTAACACTATATGCACC
				TTACATAGTAATACATTAATATTTAATACTTTTTATTTTAACTTTTTAGTTTAAAATATTATTAT
				ATTATTAACTTTTTAGTTTAAAATATTTATATTATTATAAAGAGAAATAAACAAAGGATGTTATG
				ATATATTAACACTATATGTACCTTACATAGTAATATATTAATATTTAATACTTTTTATTTTAACT
				TTTTAATTTAAAATATTATTATAAATGATGCTTGTGTTTTATGTGTTGGCATGCTTGTATTTTAT
				GTGTTGACTTTCTGTGTGAAGGTAATGTGATATGGTTAGCTGGTGGTAACAATTGTGTTTTATGT
				GTTGGCTTTCTGTGAAGCTAATTTGATATGGTTAGCTGATGGGAACAAAATATTAAAGGAAGCTA
				ATTTGATATGGTTAGCTGATAGTAACAAAATATCAAAATAAATTTCTTCTTACTTTAATAAATTA
				TATGAATTGTGACGGATTATATGGAATGTATAGGACAAAATCTTTAATAAATTACATGAATTGTG
				ACGGATTATGGAATGGAATGTAGCAAATAGGACAAAACAAATGTTTGTAAGAACCAAGAGATCCT
				AACCATGTATAGGCTAACCATATATAGGCTTAGGCCAAAAACAAACCTAAACTCTTAAACTTTGA
				TTTATTATAAAAAAAAAATGATTATTTAATATATAGGCTTAGGCCAAAAACAAACCTAAACTCTT
				AAACTTTGATTTATTATAAAAAAAAAATGATTATTTAACACCATGCACCTTACACCCTACTAAAT
				TTTAATATTTTAATTATTTTTAATTTTAAAAATTATTTTAATATTTTACATTATTTTCACTAAAA
				ACATTTGTTTTTTTTATTATAAATATTAATATTTTATTATAGATATAAAGTGAAGCAATATTGTT
				TGAAAATATCATTACCCGTTACAAAATATTGTACAACTAGCTATAAAAAAGCAAACCACAAGGAA
				ACAGAAGACTTTCACTTTGAAAAGGGGTGCCTGCTAAGACCGTAATTTTGCTTCAGATAAAAAAC
				CATCACAACTCAATAGGTACTCACAAAACTACTT
			!
			\vspace*{4pt}
		\end{minipage}
	\\
	\usename{TBS} & 
		\begin{minipage}{.85\textwidth}
			\vspace*{4pt}
			\DNA!
				TTCCTAACACCTGGAGAACCTTTTATGTACTTCACAACCCTCAATGCTGCTTCCAGGTGTGACTG
				TTTGGGTTGCTGCATGAATTGACTCAGCACCTGCACTGCAAAGCTGATATCTGGCCTTGTGATTG
				TCAAGTAGAGAAGCTTTCCCACCAGTCTCTGATAGGATCCCACATCCTCTAACTCTGCATCATCA
				GTCTTTCCTAAGTGCTTGTCATATTCTACAGTAGTCAGCCTTTGGTTCTGTTCCATTGGGGTTGA
				CACTGGTCTGCAGCCTCCCAGACCAACACCTGATATCAGTTCCAGTGCATACTTCCTCTGGTTCA
				GTAGGATTCCCTTTTCTGATCTCAGCACCTCAATGCCTAAGAAGTATTTTAGTTCTCCCAAATCT
				TTCATCTTGAAATGCTGATGCAGGGTTGCCTTTGCTTCTGAAATCAAAACATTGCTGCTGCCTGT
				TATTAACAGATCATCCACATAAATCAGGATTATGACAAGGTCAGTCCCCTCTCTTTTGGTAAACA
				AGGAGTGATCATAAGCACTTTGCATAAAACCAGCCTGCATAAGGACAGTGGTAAGCTTGATGTTC
				CACTGCCTTGATGCTTGCTTTAAACCATAGAGGGATTCAACAGCCTGCACACTTTGGACTCCCCT
				TGGCTGTGAAAACCCTGAGGCAGAGACATATAAACTTCTTCCATGAGGTCACCTTGTAGAAAAGC
				ATTGTTGACATCCATCTGGAAAAGGAACCAGCCCTTGGAAGCAGCAACAGATATGACAGCTTTTA
				CAGTGACCATTTTGGCCACTGGAGAAAAAGTTTCATGGTAGTCAAGGCCTTCTTGCTGAGTGTAT
				CCCTTGGCCACTAGCCTTGCCTTAAACCTGTCAACTTCACCATTAGCTTTGTATTTAATTTTGTA
				CACCCATTTGGACCCTATAGGCTGTTTACCAGGGGGTAAAGGGACAATCTCCCAGGTGTTATTAT
				CCTCAAGAGCCTGTATCTCAAGGGACATGGCCTCCATCCATTTCTCATCTTGAGCTGCTTCTTTG
				AAGGATTTAGGTTCAGTATAAGTGGAGAAAGCACTCAAATAAGCTTGATAGTGAACAGGCAAGTG
				ATCATAGGAAACATAGTTGGCTATAGGGTATGGAACATCCCTAGAGCCTTTGTTCAGTGTCACAA
				AGTCCTTGAGCCAGATGGGAGGACCTGCATTGCGTTTAGGTCTAGTGTGCAGGTTTGCTGGAACT
				AGAGAGGGATCAGCTACAGCAGTATGGTGCTCAAATTCAGCATTTGCTAGGTCAGGCTCAGCTGA
				CTCAACTGACTCTGCAGGAGCATGCAGGTGGCCTGAAGGTGCAGCATCAGCTGAAGTGATATGAT
				TTTGGGTAGAATGAGTCCCCAAAGTGATATCCTCATTGGCCTCTACAATGTCAGGAATAAGAACT
				GGAGTATCATCATCATCAGTATGAAAGCTGGAAGGAAAAATATCATTATACACTAGCTGCAAAGC
				TGTATCTTCAGAGCTAAGTGAACCTGCTCGAGTGAACATATCAGGTTCATGGGAGATGGACTCTT
				TGAAAGGGAACTGAAACTCTCTGAAAACTACATCCCTGCTCACTATGATCACCTTATTATCCAAG
				TCATACAACCTGTAACCCTTTTGAGTCTCAGAATAACCAATGAAGATGGTTCGCTTGGCTCTAGG
				TGCAAGTTTGTCACCTTTGGGCAGTGTTGCTGCAAAAGCAAGACACCCAAACACTCTCAAGTGAT
				CAAGCTTGGCATGTTTCTGATAGAGTAGTTCATATGGACATTTGCCTTGTAAGATTGGAGTAGGG
				AGTCTATTGATCAGGTATACAACAGTCTTGACACAGTCTCCCCAAAACCTGGTAGGTACACCACT
				CTGAAACTTAAGTGCCCTTGCCATCTCAAGGATGTGTCTGTGCTTTCTCTCCACAACACCATTCT
				GTTGTGGTGTGTAGGGACAGCTACTTTGATGAACAATCCCAAGAGAGGCCAGCAACTCATTACAA
				CTT
			!
			\vspace*{4pt}
		\end{minipage}
	\\
	\rowcolor{gray!10}\usename{lambda-EXOB} & 
		\begin{minipage}{.85\textwidth}
			\vspace*{4pt}
			\DNA!'{gray!10}
				AATTCAAACAGGGTTCTGGCGTCGTTCTCGTACTGTTTTCCCCAGGCCAGTGCTTTAGCGTTAAC
				TTCCGGAGCCACACCGGTGCAAACCTCAGCAAGCAGGGTGTGGAAGTAGGACATTTTCATGTCAG
				GCCACTTCTTTCCGGAGCGGGGTTTTGCTATCACGTTGTGAACTTCTGAAGCGGTGATGACGCCG
				AGCCGTAATTTGTGCCACGCATCATCCCCCTGTTCGACAGCTCTCACATCGATCCCGGTACGCTG
				CAGGATAATGTCCGGTGTCATGCTGCCACCTTCTGCTCTGCGGCTTTCTGTTTCAGGAATCCAAG
				AGCTTTTACTGCTTCGGCCTGTGTCAGTTCTGACGATGCACGAATGTCGCGGCGAAATATCTGGG
				AACAGAGCGGCAATAAGTCGTCATCCCATGTTTTATCCAGGGCGATCAGCAGAGTGTTAATCTCC
				TGCATGGTTTCATCGTTAACCGGAGTGATGTCGCGTTCCGGCTGACGTTCTGCAGTGTATGCAGT
				ATTTTCGACAATGCGCTCGGCTTCATCCTTGTCATAGATACCAGCAAATCCGAAGGCCAGACGGG
				CACACTGAATCATGGCTTTATGACGTAACATCCGTTTGGGATGCGACTGCCACGGCCCCGTGATT
				TCTCTGCCTTCGCGAGTTTTGAATGGTTCGCGGCGGCATTCATCCATCCATTCGGTAACGCAGAT
				CGGATGATTACGGTCCTTGCGGTAAATCCGGCATGTACAGGATTCATTGTCCTGCTCAAAGTCCA
				TGCCATCAAACTGCTGGTTTTCATTGATGATGCGGGACCAGCCATCAACGCCCACCACCGGAACG
				ATGCCATTCTGCTTATCAGGAAAGGCGTAAATTTCTTTCGTCCACGGATTAAGGCCGTACTGGTT
				GGCAACGATCAGTAATGCGATGAACTGCGCATCGCTGGCATCACCTTTAAATGCCGTCTGGCGAA
				GAGTGGTGATCAGTTCCTGTGGG
			!
			\vspace*{4pt}
		\end{minipage}
	\\
	\usename{BEAD-1C} & 
		\begin{minipage}{.85\textwidth}
			\vspace*{4pt}
			\DNA!
				TTCAGTAATACGGGTAGCTGGGACATGCCATATTTGGAACACATTTATACTAAAAAAGTATTCAT
				TGTTTATCTGAAATTCAAATTCCACTGGGCATCCTGTGTTTTATCTGGCAATGCTAGGCATGCAG
				AATACCAAAAGTAAGCACCAGGCAGGCCAGAGTCCCACCATGAGCATCTTCAGGGCCCCTGGATG
				TGGAAGAGGGATGTTGAGGGCCCAGGGGCTGCCTTGCCGGTGCATTGGCTGCCCAGGCCTGCACT
				GCCGCCTGCCGGCAGGGGTCCAGTCCACGAGTCCCAGCTCCCTGCTGGCGGAAGTCCATTTCAGA
				GCTTCCGGTTCTCCCAAGTCCAAGGATTATGCTCACTCCCCACCCACAGTCTCTTAGTGTCTGTC
				CCTGCTCTAAAGATGTTGTCTGGGCTTGATATTAATATGAGAGCTGACTGTTCCCTTCCTGATCT
				AGACCATAACCATCTTCAAGTTAAATTGCTCCTCCTCTTCTAACTGCCCAACCTCACCCACGTCT
				GACCATACCCAAGCACAG
			!
			\vspace*{4pt}
		\end{minipage}
	\\
	\rowcolor{gray!10}\usename{UASrpg} & 
		\begin{minipage}{.85\textwidth}
			\vspace*{4pt}
			\DNA!'{gray!10}
				AGCTTGCCTCGTCCCGCCGGGTCACCCGGCCAGCGACATGGAGGCCCAGAATACCCTCCTTGACA
				GTCTTGACGTGCGCAGCTCAGGGGCATGATGTGACTGTCGCCCGTACATTTAGCCCATACATCCC
				CATGTATAATCATTTGCATCCATACATTTTGATGGCCGCACGGCGCGAACGAAAAATTACGGCTC
				CTCGCTGCAGACCTGCGAGCAGGGAAACGCTCCCCTCACAGACGCGTTGAATTCTCCCCACGGCG
				CGCCCCTGTAGAGAAATATAAAAGGTTAGGATTTGCCACTGAGGTTCTTCTTTCATATACTTCCT
				TTTAAAATCTTGCTAGGATACAGTTCTCACATCACATCCGAACATAAACAAAA
			!
			\vspace*{4pt}
		\end{minipage}
	\\
	\usename{gypsy} & 
		\begin{minipage}{.85\textwidth}
			\vspace*{4pt}
			\DNA!
				CACGTAATAAGTGTGCGTTGAATTTATTCGCAAAAACATTGCATATTTTCGGCAAAGTAAAATTT
				TGTTGCATACCTTATCAAAAAATAAGTGCTGCATACTTTTTAGAGAAACCAAATAATTTTTTATT
				GCATACCCGTTTTTAATAAAATACATTGCATACCCTCTTTTAATAAAAAATATTGCATACTTTGA
				CGAAACAAATTTTCGTTGCATACCCAATAAAAGATTATTATATTGCATACCCGTTTTTAATAAAA
				TACATTGCATACCCTCTTTTAATAAAAAATATTGCATACGTTGACGAAACAAATTTTCGTTGCAT
				ACCCAATAAAAGATTATTATATTGCATACCTTTTCTTGCCATACCATTTAGCCGATCAATT
			!
			\vspace*{4pt}
		\end{minipage}
	\\
	\rowcolor{gray!10}\usename{sIns1} & 
		\begin{minipage}{.85\textwidth}
			\vspace*{4pt}
			\DNA!'{gray!10}
				AGAACTTTAAAAGTGCTCATCATTGGAAAACGTTCTTCGGGGCGAAAACTCTCAAGGATCTTACC
				GCTGTTGAGATCCAGTTCGATGTAACCCACTCGTGCACCCAACTGATCTTCAGCATCTTTTACTT
				TCACCAGCGTTTCTGGGTGAGCAAAAACAGGAAGGCAAAATGCCGCAAAAAAGGGAATAAGGGCG
				ACACGGAAATGTTGAATACTCATACTCTTCCTTTTTCAATATTATTGAAGCATTTATCAGGGTTA
				TTGTCTCATGAGCGGATACATATTTGAATGTATTTAGAAAAATAAACAAATAGGGGTTCCGCGCA
				CATTTCCCCGAAAAGTGCCACCTGACGTCCAACATATGGCACCGGAGGCTTTCGTCTTCAC
			!
			\vspace*{4pt}
		\end{minipage}
	\\
	\usename{sIns2} & 
		\begin{minipage}{.85\textwidth}
			\vspace*{4pt}
			\DNA!
				GCTGAACGCAAAGCTGATCACTCAGCGGAAGTTCGACAATCTCACTAAGGCTGAGAGGGGCGGAC
				TGAGCGAACTGGACAAAGCAGGATTCATTAAACGGCAACTTGTGGAGACTCGGCAGATTACTAAA
				CATGTCGCCCAAATCCTTGACTCACGCATGAATACCAAGTACGACGAAAACGACAAACTTATCCG
				CGAGGTGAAGGTGATTACCCTGAAGTCCAAGCTGGTCAGCGATTTCAGAAAGGACTTTCAATTCT
				ACAAAGTGCGGGAGATCAATAACTATCATCATGCTCATGACGCATATCTGAATGCCGTGGTGGGA
				ACCGCCCTGATCAAGAAGTACCCAGCACTGGAAAGCGAGTTCGTGTACGGAGACTACAAGGTCTA
				CGACGTGCGCAAGATGATTGCCAAATCTGAGCAGGAGATCGGAAAGGCCACCGCAAAGTACTTCT
				TCTACAGCAACATCATGAATTTCTTCAAGACCGAAATCACCCTTGCAAA
			!
			\vspace*{4pt}
		\end{minipage}
	\\
\end{xltabular}%
		\end{longstab}
	
	\fi
	%%% Supplemenary tables end
	
	%%% TOC icon start
	\ifTOCicon
		
		\begin{TOCicon}
			\tikzset{jpeg export = 600}%
			\begin{tikzpicture}
	\setlength{\templength}{4.5cm}

	%%% constructs strong enhancer
	\coordinate[yshift = -\columnsep] (constructs strong) at (0, 0);
	
	% insulator construct
	\coordinate (construct) at (constructs strong);
	
	\draw[line width = .25cm, -Triangle Cap, 35Senhancer]  ($(construct) + (.2, 0)$) -- ++(.75, 0) coordinate[pos = .5] (enh);

	\fill[insulator] (construct) ++(1, -.2) -- ++(60:.6) coordinate (ins) -- ++(-60:.6) -- cycle;

	\draw[line width = .25cm, 35Spromoter] (construct) ++(1.65, 0) -- ++(.4, 0) coordinate[pos = .5] (pro);
	\draw[-{Stealth[round]}, semithick] (pro) ++(.2cm + .5\pgflinewidth, 0) |- ++(.4, .35);
	
	\node[draw = black, thin, anchor = west, minimum width = 1cm, outer xsep = 0pt, font = \normalsize] (GFP) at ($(pro) + (.45, 0)$) {~GFP};
	
	\begin{pgfonlayer} {background}
		\fill[Orchid1] (GFP.north west) ++(.15, 0) rectangle ($(GFP.south west) + (.05, 0)$);
	\end{pgfonlayer}
	
	\coordinate[xshift = .2cm] (construct end) at (GFP.east);
	
	\begin{pgfonlayer} {background}
		\draw[thick, {Bar[width = .1cm]}-] (construct) -- (GFP.west);
		\draw[thick, -{Bar[width = .1cm]}] (GFP.east) -- (construct end);
	\end{pgfonlayer}
	
	\draw[35Senhancer, -{Stealth[round]}, dashed, thick] (enh) ++(0, .15) to[out = 60, in = 120] ($(pro) + (0, .15)$);
	
	\draw[insulator, ultra thick] (ins) ++(0, .125) +(-.125, .125) -- +(.125, -.125) +(-.125, -.125) -- +(.125, .125);
	
	\coordinate (construct center) at ($(construct)!.5!(construct end)$);
	
	% activity
	\node[anchor = north west, font = \LARGE\bfseries, align = right, shift = {(-.33, -.33)}] (ins act) at (construct) {\color{insulator}insulator\\[-.25\baselineskip]activity};
	\node[right = 0pt of ins act, node font = \fontsize{50}{60}\selectfont\fontspec{Wingdings}, Green3, inner sep = 0pt, yshift = -.1cm] {\char61692};
	
	% silencer construct
	\coordinate[yshift = -2.5cm] (construct 2) at (construct);

	\fill[insulator] (construct 2) ++(.15, -.2) -- ++(60:.6) coordinate (ins) -- ++(-60:.6) -- cycle;
	
	\draw[line width = .25cm, -Triangle Cap, 35Senhancer]  ($(construct 2) + (.8, 0)$) -- ++(.75, 0) coordinate[pos = .5] (enh);

	\draw[line width = .25cm, 35Spromoter] (construct 2) ++(1.65, 0) -- ++(.4, 0) coordinate[pos = .5] (pro);
	\draw[-{Stealth[round]}, semithick] (pro) ++(.2cm + .5\pgflinewidth, 0) |- ++(.4, .35);
	
	\node[draw = black, thin, anchor = west, minimum width = 1cm, outer xsep = 0pt, font = \normalsize] (GFP) at ($(pro) + (.45, 0)$) {~GFP};
	
	\begin{pgfonlayer} {background}
		\fill[MediumPurple2] (GFP.north west) ++(.15, 0) rectangle ($(GFP.south west) + (.05, 0)$);
	\end{pgfonlayer}
	
	\coordinate[xshift = .2cm] (construct 2 end) at (GFP.east);
	
	\begin{pgfonlayer} {background}
		\draw[thick, {Bar[width = .1cm]}-] (construct 2) -- (GFP.west);
		\draw[thick, -{Bar[width = .1cm]}] (GFP.east) -- (construct 2 end);
	\end{pgfonlayer}
	
	\draw[35Senhancer, very thick, -{Stealth[round]}] (enh) ++(0, .15) to[out = 60, in = 120] ($(pro) + (0, .15)$);
	\draw[35Senhancer, very thick, line cap = round] ($(enh) + (.175, .525)$) -- ++(0, -.15) ++(-.075, .075) -- ++(.15, 0) ++(-.075, 0) coordinate (plus);
	\draw[35Senhancer] (plus) circle (.12);
	
	\draw[silencer, thick, -|, dashed] (pro -| ins) ++(0, -.25) to[out = -45, in = 270, looseness = .75] ($(pro) + (0, -.15)$);
	
	% activity
	\node[anchor = north east, font = \LARGE\bfseries, align = right, shift = {(0, -.45)}] (sil act) at (ins act.east |- construct 2) {\color{insulator}silencer\\[-.25\baselineskip]activity};
	\node[right = 0pt of sil act, node font = \fontsize{50}{60}\selectfont\fontspec{Wingdings}, Firebrick2, inner sep = 0pt, yshift = -.1cm] {\char61691};
	
	% enhancer label
	\node[anchor = south, node font = \huge\bfseries, align = center, yshift = .75cm] (lstrong) at (construct center) {strong\\[-.25\baselineskip]\color{35Senhancer}enhancer};
	
	
	%%% constructs weak enhancer
	\coordinate[xshift = \templength] (constructs weak) at (constructs strong);
	
	% insulator construct
	\coordinate (construct 3) at (constructs weak);
	
	\draw[line width = .25cm, -Triangle Cap, 35Senhancer!50]  ($(construct 3) + (.2, 0)$) -- ++(.75, 0) coordinate[pos = .5] (enh);

	\fill[insulator] (construct 3) ++(1, -.2) -- ++(60:.6) coordinate (ins) -- ++(-60:.6) -- cycle;

	\draw[line width = .25cm, 35Spromoter] (construct 3) ++(1.65, 0) -- ++(.4, 0) coordinate[pos = .5] (pro);
	\draw[-{Stealth[round]}, semithick] (pro) ++(.2cm + .5\pgflinewidth, 0) |- ++(.4, .35);
	
	\node[draw = black, thin, anchor = west, minimum width = 1cm, outer xsep = 0pt, font = \normalsize] (GFP) at ($(pro) + (.45, 0)$) {~GFP};
	
	\begin{pgfonlayer} {background}
		\fill[HotPink1] (GFP.north west) ++(.15, 0) rectangle ($(GFP.south west) + (.05, 0)$);
	\end{pgfonlayer}
	
	\coordinate[xshift = .2cm] (construct 3 end) at (GFP.east);
	
	\begin{pgfonlayer} {background}
		\draw[thick, {Bar[width = .1cm]}-] (construct 3) -- (GFP.west);
		\draw[thick, -{Bar[width = .1cm]}] (GFP.east) -- (construct 3 end);
	\end{pgfonlayer}
	
	\draw[35Senhancer!50, -{Stealth[round]}, dashed, thick] (enh) ++(0, .15) to[out = 60, in = 120] ($(pro) + (0, .15)$);
	
	\draw[insulator, ultra thick] (ins) ++(0, .125) +(-.125, .125) -- +(.125, -.125) +(-.125, -.125) -- +(.125, .125);
	
	% activity
	\node[anchor = north west, font = \LARGE\bfseries, align = right, shift = {(-.33, -.33)}] (ins act) at (construct 3) {\color{insulator}insulator\\[-.25\baselineskip]activity};
	\node[right = 0pt of ins act, node font = \fontsize{50}{60}\selectfont\fontspec{Wingdings}, Green3, inner sep = 0pt, yshift = -.1cm] {\char61692};
	
	
	% silencer construct
	\coordinate (construct 4) at (construct 3 |- construct 2);

	\fill[insulator] (construct 4) ++(.15, -.2) -- ++(60:.6) coordinate (ins) -- ++(-60:.6) -- cycle;
	
	\draw[line width = .25cm, -Triangle Cap, 35Senhancer!50]  ($(construct 4) + (.8, 0)$) -- ++(.75, 0) coordinate[pos = .5] (enh);

	\draw[line width = .25cm, 35Spromoter] (construct 4) ++(1.65, 0) -- ++(.4, 0) coordinate[pos = .5] (pro);
	\draw[-{Stealth[round]}, semithick] (pro) ++(.2cm + .5\pgflinewidth, 0) |- ++(.4, .35);
	
	\node[draw = black, thin, anchor = west, minimum width = 1cm, outer xsep = 0pt, font = \normalsize] (GFP) at ($(pro) + (.45, 0)$) {~GFP};
	
	\begin{pgfonlayer} {background}
		\fill[Purple1] (GFP.north west) ++(.15, 0) rectangle ($(GFP.south west) + (.05, 0)$);
	\end{pgfonlayer}
	
	\coordinate[xshift = .2cm] (construct 4 end) at (GFP.east);
	
	\begin{pgfonlayer} {background}
		\draw[thick, {Bar[width = .1cm]}-] (construct 4) -- (GFP.west);
		\draw[thick, -{Bar[width = .1cm]}] (GFP.east) -- (construct 4 end);
	\end{pgfonlayer}
	
	\draw[35Senhancer!50, very thick, -{Stealth[round]}] (enh) ++(0, .15) to[out = 60, in = 120] ($(pro) + (0, .15)$);
	\draw[35Senhancer!50, very thick, line cap = round] ($(enh) + (.175, .525)$) -- ++(0, -.15) ++(-.075, .075) -- ++(.15, 0) ++(-.075, 0) coordinate (plus);
	\draw[35Senhancer!50] (plus) circle (.12);
	
	\draw[silencer, very thick, -|] (pro -| ins) ++(0, -.25) to[out = -45, in = 270, looseness = .75] ($(pro) + (0, -.15)$);
	\draw[silencer, very thick, line cap = round] (pro) ++(0, -.4) -- ++(.15, 0) coordinate[pos = .5] (minus);
	\draw[silencer] (minus) circle (.12);
	
	% activity
	\node[anchor = north east, font = \LARGE\bfseries, align = right, shift = {(0, -.45)}] (sil act) at (ins act.east |- construct 4) {\color{insulator}silencer\\[-.25\baselineskip]activity};
	\node[right = 0pt of sil act, node font = \fontsize{50}{60}\selectfont\fontspec{Wingdings}, Green3, inner sep = 0pt, yshift = -.1cm] {\char61692};
	
	% enhancer label
	\node[anchor = base, node font = \huge\bfseries, align = center, xshift = \templength] (lweak) at (lstrong.base) {weak\\[-.25\baselineskip]\color{35Senhancer!50}enhancer};

	
	%%% frame
	\draw[thick] (current bounding box.north west) ++(-.2, 0) coordinate (top left) -| ($(current bounding box.south east) + (.2, 0)$) coordinate[pos = .25] (top center) coordinate (bottom right) -| cycle;
	\draw[thick] (top center) -- (top center |- bottom right);
	
	\begin{pgfonlayer}{backbackground}
		\shade[top color = white, bottom color = 35Senhancer!20] (top left) rectangle (top center |- bottom right);
		\shade[top color = white, bottom color = 35Senhancer!10] (top center) rectangle (bottom right);
	\end{pgfonlayer}

\end{tikzpicture}%
		\end{TOCicon}
		
	\fi
	%%% TOC icon end
	

\end{document}